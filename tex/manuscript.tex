\documentclass[twocolumn, linenumbers, superscriptaddress]{revtex4-1}

\usepackage{graphicx}
\usepackage{blindtext}
\usepackage{gensymb}
\usepackage{xcolor}
\usepackage[colorlinks]{hyperref}
	\hypersetup{colorlinks,
	linkcolor={red!50!black},
	citecolor={blue!60!black},
	urlcolor={blue!40!black}
	}
\usepackage{cleveref}
\graphicspath{{figures/}}

\begin{document}
	%abstract - approx. 150 words
	\begin{abstract}
		\blindtext
	\end{abstract}
	
	%title and author block
	{\title{Microstructure-derived strength in the acorn weevil exoskeleton}
	
	\date{\today}
	
	\author{M. Andrew Jansen}
		\email[corresponding author, email:~]{majanse1@asu.edu}
		\affiliation{School of Life Sciences, Arizona State University, Tempe, AZ 85287, USA}
	\author{Nikhilesh Chawla}
		\affiliation{School for Engineering of Matter, Energy, and Transport, Arizona State University, Tempe, AZ 85287, USA}
	\author{Nico M. Franz}
		\affiliation{School of Life Sciences, Arizona State University, Tempe, AZ 85287, USA}
		
	\maketitle
	}
	
%introduction - nature materials requires that introduction is given NO SECTION HEADER
%Brief explanation of relevant behavior and biology, prior work on ecology, head shape, cuticle mechanics, and rostral microstructure.
	
	We report novel modifications to the composite microstructure of the exoskeleton in the snout of acorn weevils (Coleoptera: Curculionidae) belonging to the genus \textit{Curculio} Linnaeus, 1756.

	As a weevil (snout beetle), members of the genus \textit{Curculio} are typified by the presence of a highly elongate structure on the head, called the rostrum (snout). 
	This structure is a hollow, strongly curved (over 90$\degree$), cylindrical extension of the exoskeleton of the otherwise nearly-spherical head, which bears at its apex the terminal chewing mouthparts. 
	The space inside of the rostrum contains the esophagus, various muscles and tendons used for feeding, and hemolymph that serves as a rough equivalent to blood in insects.
	By contrast, the solid shell of the rostrum is comprised entirely of cuticle, which can be considered a laminate composite consisting of various arrangements of chitin fibers embedded in a protein matrix (see \cref{sec:microstructure}).
	Acorn weevils use this structure to excavate sites for egg-laying (oviposition) and feeding on a variety of fruits, including acorns, Japanese camellia, hazelnuts, pecans, chestnuts, and chinquapins. 
		
	During oviposition, a female engages in a unique "drilling" behavior that causes significant, apparently elastic, deformation of the rostrum.
	The female will insert the snout into an incision made with the mandibles, eating the material as she proceeds, while rotating her head and body around the perimeter of the bore-hole.
	Once the apex of the snout is fully inserted, she will push up and forward with her front legs, forcing the rostrum to bend until it is nearly straight.
	The female will maintain tension on the rostrum in this position, continuing to ingest the substrate and rotate around the bore-hole, while slowly inserting the rostrum further into the excavated channel.
	Once the rostrum is fully inserted, usually up to the eyes, she will pull her snout from the bore-hole and deposit several eggs into the site.
	By maintaining constant tension on the rostrum and rotating around the bore-hole, the female is able to flex the snout into a near-perfectly straight configuration and thereby produce a linear channel into the fruit.
	While this behavior has been observed in many species of \textit{Curculio}, we have lacked a fundamental understanding of how female \textit{Curculio} rostra can withstand the repeated, often extreme bending incurred during the process of oviposition.

	We have found that the composite profile of the rostrum is strongly differentiated from the head capsule and other body parts, with modification of both the relative layer thicknesses and fiber orientation angles of cuticle regions (viz. exocuticle and endocuticle), which we describe in detail below.
	We posit that these modifications enable the snout to be flexed until straight while remaining within the elastic limits of the material, mitigating the risk of structural damage, and without evident alteration of the mechanical properties of the individual components of the cuticle across the structure and between species.
	Thus, the flexibility and tensile strength of the rostrum appear to be derived \emph{exclusively} from modification of the composite architecture of the exoskeleton.
	
	Support for this hypothesis has come from three lines of evidence: 
	\begin{enumerate}
	\item Examination of the cuticle microstructure across the length of the snout has revealed consistent modification to the composite structure of the rostrum among \textit{Curculio} species.
	\item Tensile testing of the rostrum has demonstrated that the mechanical strength of the cuticle components are consistent along the length of the structure and between species.
	\item Fatigue testing has shown that a highly curved rostrum is capable of flexing hundreds of thousands of times without damage to the structure, and is apparently elastic.
	\end{enumerate}
	
	We additionally describe the fracture mechanics of the snout, as pertains to both cuticle composite structure and tensile behavior, and consider how modification of the cuticle may reduce the risk of rostral fracture during oviposition.
	To our knowledge, this is the first time that a modified composite profile has been reported as a means of enhancing structural elasticity in the insect exoskeleton.

%4 sections of results ideally, nobody really uses subsections, 6 display items max.
	\section{Microstructure of the rostrum}\label{sec:microstructure} 
		Display 1: Pictures of heads, macrofiber arrangement, and exo-endo ratio at base and apex, across species.
		Explain how everything is laid out across the length of the rostrum, emphasizing that this is key to predicting and understanding the mechanical behavior of the snout during bending.
		No change in total thickness, only relative layer thickness,
		No resilin either.
		No change in total thickness, only relative layer thickness, no resilin
		outline general arrangement of normal beetle/weevil cuticle (copy/edit from previous)
		detail specifics of head capsule cuticle 
		explain modification of rostral cuticle
	
	\section{Tensile testing and fracture mechanics}
		The predicted behavior of the snout is borne out by the data, where UTS is strongly correlated with the cross-sectional area of the endocuticle across species.
		
	\section{Fatigue testing of \textit{Curculio caryae}}

%conclusion	- with intro and "results" should be no more than 3000 words
	\section{Conclusion}

%actual methods after references
	\section{Methods}
		Methods, including statements of data availability and any associated accession codes and references, are available in the online version of this paper.
	
%50 references (as a guideline)
	\begin{thebibliography}{50}
		\section*{References}	
		\bibitem{lamport94}
			Leslie Lamport,
			\textit{\LaTeX: a document preparation system},
			Addison Wesley, Massachusetts,
			2nd edition,
			1994.

	\end{thebibliography}

%brief and 'without effusive statements'
%thank everyone who helped with data collection, especially jason (tensile testing), sridhar (if I use micro-CT scan data), dave (SEM), charlie (specimens), paige (if I use confocal data), Sal (if I use histology/staining data or confocal data, he helped with specimen clearing).
%Financial support/ funding sources/ grants.
	\begin{acknowledgements}

	\end{acknowledgements}

%author contributions
%MAJ: did tensile tests, SEM, confocal (if applicable), data analysis.
%MAJ, NC: designed research, micro-CT (if applicable)
%MAJ, NMF: field collection of test specimens, procurement of specimen loans, specimen prep.
%MAJ, NC, NMF: All authors contributed to the writing, review, and revision of the manuscript.
	\section*{Author contributions}
		\begin{description}
		\item[Andrew Jansen] Conducted sectioning and staining, microscopy and imaging, tensile and fatigue testing, statistical analysis, and participated in manuscript preparation.
		\item[Nikhilesh Chawla] Facilitated microscopy, tensile and fatigue testing, and participated in manuscript preparation.
		\item[Nico Franz] Facilitated specimen acquisition and imaging, participated in manuscript preparation.
		\end{description} 
	
	\section*{Additional information}
		Supplementary information is available in the online version of the paper.
		Reprints and permissions information is available online at www.nature.com/reprints.
		Correspondence and requests for materials should be addressed to M.A.J.
	
	\section*{Competing financial interests}
		The authors declare no competing financial interests.
	
	\newpage

%provide detailed methods here, subsection for each method, statistical analysis, statements of code and data avilability.
	\section*{Methods}
		\subsection*{Histological sectioning}
			
		\subsection*{Tensile and fatigue testing}

		\subsection*{Specimen imaging and microscopy}
			
		\subsection*{Statistical analysis}
			\paragraph*{General approach}
				To explore the relationships between the composite structure and mechanical properties of the cuticle, we fit phylogenetic linear mixed-effects models to the data using maximum likelihood estimation.
				In order to control for phylogenetic non-independence in the data, we included the species of each specimen as a random effect in all models.
				We also allowed for correlation in the error term of the models, as specified by a variance-covariance matrix generated from a Brownian motion model of trait evolution along the phylogeny.
				Response variables and covariates were natural-log transformed, as needed, to ensure model residuals were normally distributed and homoscedastic.
				In all models, we tested whether the inclusion of phylogenetic correlation in the model error produced significantly better model fit, using a likelihood-ratio test and $R^{2}_{\sigma}$-difference test between the fully-specified model and a model lacking the phylogenetic effect.
			\paragraph*{Hypothesis testing}
				The following three hypotheses were tested using PGLMMs fitted using ML estimation:
				\begin{enumerate}
				\item The maximum sustained tensile force is proportional to the cross-sectional area of the endocuticle, and \emph{not} that of the exocuticle.
				\item The ultimate tensile strength of the samples is inversely proportional to the ratio of exocuticle to endocuticle at the location of fracture.
				\item Young's modulus of the samples is proportional to the length of the snout.
				\end{enumerate}
				
				We fitted a fully-specified model with the cross-sectional area of endocuticle and exocuticle at the site of fracture as fixed effects, including an interaction term, and with maximum tensile force sustained prior to fracture as a response variable.
				This model was then compared to models with only cross-sectional area of either endocuticle or exocuticle as the sole fixed effect in the model.
				We then tested the first hypothesis by using likelihood-ratio tests and $R^{2}_{\beta*}$-difference tests between each of the three models.
				
				The hypothesis that 
			
 
			\paragraph*{Model selection and fitting}
			\paragraph*{Estimating phylogenetic signal}
		
			
		\subsection*{Code availability}
			
		\subsection*{Data availability}

%references for methods only
	\begin{thebibliography}{50}
		\section*{References}	
		\bibitem{lamport94}
			Leslie Lamport,
			\textit{\LaTeX: a document preparation system},
			Addison Wesley, Massachusetts,
			2nd edition,
			1994.

	\end{thebibliography}
\end{document}