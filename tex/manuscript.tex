\documentclass[twocolumn, linenumbers, superscriptaddress]{revtex4-1}

\usepackage{graphicx}
\usepackage{blindtext}
\usepackage{gensymb}
\usepackage{xcolor}
\usepackage[colorlinks]{hyperref}
	\hypersetup{colorlinks,
	linkcolor={red!50!black},
	citecolor={blue!60!black},
	urlcolor={blue!40!black}
	}
\usepackage{cleveref}
\graphicspath{{figures/}}

\begin{document}
	%abstract - approx. 150 words
	\begin{abstract}
		\blindtext
	\end{abstract}
	
	%title and author block
	{\title{Microstructure-derived strength in the acorn weevil exoskeleton}
	
	\date{\today}
	
	\author{M. Andrew Jansen}
		\email[corresponding author, email:~]{majanse1@asu.edu}
		\affiliation{School of Life Sciences, Arizona State University, Tempe, AZ 85287, USA}
	\author{Jason Williams}
		\affiliation{School for Engineering of Matter, Energy, and Transport, Arizona State University, Tempe, AZ 85287, USA}
	\author{Nikhilesh Chawla}
		\affiliation{School for Engineering of Matter, Energy, and Transport, Arizona State University, Tempe, AZ 85287, USA}
	\author{Nico M. Franz}
		\affiliation{School of Life Sciences, Arizona State University, Tempe, AZ 85287, USA}
		
	\maketitle
	}
	
%introduction - nature materials requires that introduction is given NO SECTION HEADER
%Brief explanation of relevant behavior and biology, prior work on ecology, head shape, cuticle mechanics, and rostral microstructure.

%650 words here

	We report novel modifications to the composite microstructure of the exoskeleton in the snout of acorn weevils (Coleoptera: Curculionidae) belonging to the genus \textit{Curculio} Linnaeus, 1756.

	As a weevil (snout beetle), members of the genus \textit{Curculio} are typified by the presence of a highly elongate structure on the head, called the rostrum (snout). 
	This structure is a hollow, strongly curved (over 90$\degree$), cylindrical extension of the exoskeleton of the otherwise nearly-spherical head, which bears at its apex the terminal chewing mouthparts. 
	The space inside of the rostrum contains the esophagus, various muscles and tendons used for feeding, and hemolymph that serves as a rough equivalent to blood in insects.
	By contrast, the solid shell of the rostrum is comprised entirely of cuticle, which can be considered a laminate composite consisting of various arrangements of chitin fibers embedded in a protein matrix (see \cref{sec:microstructure}).
	Acorn weevils use this structure to excavate sites for egg-laying (oviposition) and feeding on a variety of fruits, including acorns, Japanese camellia, hazelnuts, pecans, chestnuts, and chinquapins. 
		
	During oviposition, a female engages in a unique "drilling" behavior that causes significant, apparently elastic, deformation of the rostrum.
	The female will insert the snout into an incision made with the mandibles, eating the material as she proceeds, while rotating her head and body around the perimeter of the bore-hole.
	Once the apex of the snout is fully inserted, she will push up and forward with her front legs, forcing the rostrum to bend until it is nearly straight.
	The female will maintain tension on the rostrum in this position, continuing to ingest the substrate and rotate around the bore-hole, while slowly inserting the rostrum further into the excavated channel.
	Once the rostrum is fully inserted, usually up to the eyes, she will pull her snout from the bore-hole and deposit several eggs into the site.
	By maintaining constant tension on the rostrum and rotating around the bore-hole, the female is able to flex the snout into a near-perfectly straight configuration and thereby produce a linear channel into the fruit.
	While this behavior has been observed in many species of \textit{Curculio}, we have lacked a fundamental understanding of how female \textit{Curculio} rostra can withstand the repeated, often extreme bending incurred during the process of oviposition.

	We have found that the composite profile of the rostrum is strongly differentiated from the head capsule and other body parts, with modification of both the relative layer thicknesses and fiber orientation angles of cuticle regions (viz. exocuticle and endocuticle), which we describe in detail below.
	We posit that these modifications enable the snout to be flexed until straight while remaining within the elastic limits of the material, mitigating the risk of structural damage, and without evident alteration of the mechanical properties of the individual components of the cuticle across the structure and between species.
	Thus, the flexibility and tensile strength of the rostrum appear to be derived \emph{exclusively} from modification of the composite architecture of the exoskeleton.
	
	Support for this hypothesis has come from three lines of evidence:
	
	\begin{enumerate}
	\item Examination of the cuticle microstructure across the length of the snout has revealed consistent modification to the composite structure of the rostrum among \textit{Curculio} species.
	\item Tensile testing of the rostrum has demonstrated that the mechanical strength of the cuticle components are consistent along the length of the structure and between species.
	\item Fatigue testing has shown that a highly curved rostrum is capable of flexing hundreds of thousands of times without damage to the structure, and is apparently elastic.
	\end{enumerate}
	
	We additionally describe the fracture mechanics of the snout, as pertains to both cuticle composite structure and tensile behavior, and consider how modification of the cuticle may reduce the risk of rostral fracture during oviposition.
	To our knowledge, this is the first time that a modified composite profile has been reported as a means of enhancing structural elasticity in the insect exoskeleton.


	MIGHT BE GOOD TO SIGNPOST HERE... we discuss each of these lines of evidence in the sections below beginning with a consideration of the impact of microstructure, then we talk about mechanical testing, blah blah. Emphasize that this is key to predicting and understanding the mechanical behavior of the snout during bending.
%stick to 600 words for each results section and 500 words for conclusions
	\section{Microstructure of the rostrum}\label{sec:microstructure}
	%800 words here (1500 total, shorten later)
		The cuticle of arthropods, including insects, is made up of chitin and numerous uncharacterized proteins, which effectively act as a matrix into which the chitin is embedded \cite{Nikolov2011,Nikolov2010,Vincent2004}.
		The cuticle is strongly influenced by its chemical properties, especially the degree of tanning and overall water content, but the arrangement of the embedded chitin fibers is primarily responsible for the its unique mechanical behavior \cite{Klocke2011,Vincent2004}.	
		Accordingly, the cuticle is often considered as a fiber-reinforced composite material; however, in beetles, the composite arrangement of the chitin fibers varies by cuticle region.	
	
		The insect cuticle can generally be divided into three regions (Figs.~INSERT-REFS-HERE), including (1) endocuticle, which is the most compliant and innermost region; and (3) exocuticle, which is the outermost region, usually the stiffest and hardest, and coated by a waxy secretion (epicuticle) that is not of great mechanical significance in most groups.
		Between these regions lies (2) the mesocuticle, which is similar in microstructure to exocuticle, but exhibits a lesser degree of tanning (sclerotization) and usually serves as a transition zone to the endocuticle \cite{Klocke2011,Vincent1982,Vincent2004}.
		
		In general, these regions are confluent, and not necessarily sharply defined; for instance, the mesocuticle is not always evident, and is sometimes considered part of the exocuticle.
		In beetles, both the exocuticle and mesocuticle (when present) are laminate, and have numerous laminae of unidirectional chitin fibers, one layer thick (2-4nm each), embedded in a proteinaceous matrix.
		These layers are stacked at a more or less constant angle to each other in a helicoidal arrangement, referred to as the Bouligand structure \cite{Blackwell1980,Bouligand1972,Neville1976}. 
		This microstructural arrangement allows the cuticle to exhibit transverse isotropy, or complete isotropy in some cases, despite the strong anisotropy of $\alpha$-chitin.
		
		Conversely, the endocuticle of Coleoptera appears to take advantage of this anisotropy to improve fracture resistance of the exoskeleton.
		Beetles in particular have a highly modified endocuticle made up of large (5-20$\mu$m OD) unidirectional bundles of chitin, called ``balken'' (German for beam, strut) or macrofibers; these macrofibers are aligned in layers, as in Figs~INSERT-FIGS-HERE \cite{Kamp2010,Kamp2015}.
		Typically, the endocuticle contains several layers of macrofibers, in which adjacent layers form pairs. 
		The layers within each pair are pseudo-orthogonal (i.e., angled approx. $90\degree$ to each other, see Figs.~INSERT-REFS-HERE), while the stacking angle between pairs is typically much shallower, although some other configurations have been observed \cite{Hepburn1973,Kamp2010}.
		It is thought that this arrangement of the macrofibers contributes to the high toughness of beetle exoskeletons by inhibiting crack formation and propagation between successive layers \cite{Kamp2010,Kamp2015,Hepburn1973}.		
		
		We have found that in \textit{Curculio}, the head capsule (as in the rest of the body) exhibits this general arrangement of the endocuticle, with an angle of approximately $30\degree$ between successive pseudo-orthogonal plies.
		Additionally, the head capsule has roughly equal thicknesses of exocuticle and endocuticle along its surface in cross section; qualitatively, manipulating this part of the head with forceps quickly reveals that this part of the head is quite rigid.
		
		By contrast, the region beyond the scrobe (antennal channel) is quite flexible, even in fully desiccated specimens.
		Serial thin sectioning of the snout has demonstrated that the cuticle in this region has substantially different composite properties from the head capsule in two key traits (see Fig~INSERT-REFS-HERE):

		\begin{enumerate}
			\item The exocuticle is reduced to a thin shell, with the endocuticle thickened to offset this reduction.
			\item The endocuticular macrofibers exhibit no rotation between successive pseudo-orthogonal plies, which are all oriented at approx. $45\degree$ to the longitudinal axis of the snout.
		\end{enumerate}
		
		The cuticle of the rostral base and region near the scrobe exhibits a gradual transition between these cuticle arrangements, such that there is a smoothly discontinuous profile in the composite properties of the cuticle along an anterior-posterior gradient.		
		In previous work we identified these modifications to the composite arrangement of the cuticle, but only within the species \textit{C. longinasus} Chittenden, 1927 \cite{Jansen2016, Singh2016}.
		These same modifications to the cuticle of the rostral apex have now been uncovered in a further six phylogenetically disparate species, listed below, thus indicating that this is likely a genus-wide phenomenon.
		
		Notably, we could find no evidence of resilin, as indicated by florescent microscopy, anywhere in the cuticle of the head (including the rostrum).
		We also observed no difference in the total thickness of the cuticle between the head capsule and rostral apex, only differences in relative layer thicknesses.
		In the next section, we demonstrate that that the endocuticle does not vary in tensile strength across the head, making it unlikely that there are significant differences in sclerotization or chitin composition within the exocuticle and endocuticle that might result in a more compliant cuticle.
		We therefore hypothesize that these differences in the microstructure of the cuticle in different regions of the head are solely responsible for the observed flexibility of the snout.
	
	\section{Tensile testing and fracture mechanics} %shoot for 600 words
		The predicted behavior of the snout is borne out by the data, where UTS is strongly correlated with the cross-sectional area of the endocuticle across species.
		
	\section{Fatigue testing of \textit{Curculio caryae}} %shot for 600 words

	\section{Conclusion} %shoot for 500 words

%actual methods after references
	\section{Methods}
		Methods, including statements of data availability and any associated accession codes and references, are available in the online version of this paper.
	
%50 references (as a guideline, in order of appearance in text)
	\begin{thebibliography}{50}
		\section*{References}	
			\bibitem{Klocke2011}
				Klocke, D. \& Schmitz, H.
				Water as a major modulator of the mechanical properties of insect cuticle.
				\textit{Acta Biomater.}
				\textbf{7,}
				2935--2942
				(2011).
				
			\bibitem{Vincent1982}
				Vincent, J. F. V.
				\textit{Structural biomaterials}
				(Halsted Press,
				New York,
				1982).
				
			\bibitem{Vincent2004}
				Vincent, J. F. V. \& Wegst, U. G. K.
				Design and mechanical properties of insect cuticle.				
				\textit{Arthropod Struct. Dev.}
				\textbf{33:3,}
				187--199,
				(2004).
				
			\bibitem{Nikolov2011}
				Nikolov, S. et al.
				Robustness and optimal use of design principles of arthropod exoskeletons studied by ab initio-based multiscale simulations.
				\textit{J. Mech. Behav. Biomed. Mater.}
				\textbf{4:2,}
				129--145,
				(2011).
				
			\bibitem{Nikolov2010}
				Nikolov, S. et al.
				Revealing the design principles of high-performance biological composites using Ab initio and multiscale simulations: The example of lobster cuticle.
				\textit{Adv. Mater.}
				\textbf{22:4,}
				519--526,
				(2010).
				
			\bibitem{Blackwell1980}
				Blackwell, J. \& Weih, M.
				Structure of chitin-protein complexes: ovipositor of the ichneumon fly \textit{Megarhyssa}.
				\textit{J. Mol. Biol.}
				\textbf{137:1,}
				49--60,
				(1980).

			\bibitem{Bouligand1972}
				Bouligand, Y.
				Twisted fibrous arrangements in biological materials and cholesteric mesophases.
				\textit{Tissue Cell}
				\textbf{4:2,}
				189--217,
				(1972).
				
			\bibitem{Neville1976}
				Neville, A. C., Parry, D. A. \& Woodhead-Galloway, J.
				The chitin crystallite in arthropod cuticle.
				\textit{J. Cell Sci.}
				\textbf{21:1,}
				73--82,
				(1976).
				
			\bibitem{Cheng2009}
				Cheng, L., Wang, L., \& Karlsson, A. M.
				Mechanics-based analysis of selected features of the exoskeletal microstructure of \textit{Popillia japonica}.
				\textit{Mater. Res.}
				\textbf{24:11,}
				3253--3267,
				(2009).
				
			\bibitem{Hepburn1973}
				Hepburn, H. R. \& Ball, A.
				On the structure and mechanical properties of beetle shells.
				\textit{J. Mater. Sci.}
				\textbf{8:5,}
				618--623,
				(1973).
				
			\bibitem{Kamp2010}
				van de Kamp, T. \& Greven, H.
				On the architecture of beetle elytra.
				\textit{Entomol. Heute}
				\textbf{22,}
				191--204,
				(2010).
			
			\bibitem{Kamp2015}
				van de Kamp, T., Riedel, A. \& Greven, H.
				Micromorphology of the elytral cuticle of beetles, with an emphasis on weevils (Coleoptera : Curculionoidea)
				\textit{Arthropod Struct. Dev.}
				\textbf{45:1,}
				14--22,
				(2015).
			
	\end{thebibliography}

%brief and 'without effusive statements'
%thank everyone who helped with data collection, dave (SEM), charlie (specimens), paige (if I use confocal data), Sal (if I use histology/staining data or confocal data, he helped with specimen clearing).
%Financial support/ funding sources/ grants.
	\begin{acknowledgements}

	\end{acknowledgements}

%author contributions
%MAJ: did tensile tests, SEM, confocal (if applicable), data analysis.
%MAJ, NC: designed research, micro-CT (if applicable)
%MAJ, NMF: field collection of test specimens, procurement of specimen loans, specimen prep.
%MAJ, NC, NMF: All authors contributed to the writing, review, and revision of the manuscript.
	\section*{Author contributions}
		\begin{description}
		\item[Andrew Jansen] Conducted sectioning and staining, microscopy and imaging, tensile and fatigue testing, statistical analysis, and participated in manuscript preparation.
		\item[Jason Williams] Conducted tensile and fatigue testing, participated in manuscript preparation.
		\item[Nikhilesh Chawla] Facilitated microscopy, tensile and fatigue testing, and participated in manuscript preparation.
		\item[Nico Franz] Facilitated specimen acquisition and imaging, participated in manuscript preparation.
		\end{description} 
	
	\section*{Additional information}
		Supplementary information is available in the online version of the paper.
		Reprints and permissions information is available online at www.nature.com/reprints.
		Correspondence and requests for materials should be addressed to M.A.J.
	
	\section*{Competing financial interests}
		The authors declare no competing financial interests.
	
	\newpage

%provide detailed methods here, subsection for each method, statistical analysis, statements of code and data avilability.
	\section*{Methods}
		\subsection*{Histological sectioning}
			
		\subsection*{Tensile and fatigue testing}

		\subsection*{Specimen imaging and microscopy}
			
		\subsection*{Statistical analysis}
			\paragraph*{General approach}
				To explore the relationships between the composite structure and mechanical properties of the cuticle, we fit phylogenetic linear mixed-effects models to the data using maximum likelihood estimation.
				In order to control for phylogenetic non-independence in the data, we included the species of each specimen as a random effect in all models.
				We also allowed for correlation in the error term of the models, as specified by a variance-covariance matrix generated from a Brownian motion model of trait evolution along the phylogeny.
				Response variables and covariates were natural-log transformed, as needed, to ensure model residuals were normally distributed and homoscedastic.
				In all models, we tested whether the inclusion of phylogenetic correlation in the model error produced significantly better model fit, using a likelihood-ratio test and $R^{2}_{\sigma}$-difference test between the fully-specified model and a model lacking the phylogenetic effect.
			\paragraph*{Hypothesis testing}
				The following three hypotheses were tested using PGLMMs fitted using ML estimation:
				\begin{enumerate}
				\item The maximum sustained tensile force is proportional to the cross-sectional area of the endocuticle, and \emph{not} that of the exocuticle.
				\item The ultimate tensile strength of the samples is inversely proportional to the ratio of exocuticle to endocuticle at the location of fracture.
				\item Young's modulus of the samples is proportional to the length of the snout.
				\end{enumerate}
				
				We fitted a fully-specified model with the cross-sectional area of endocuticle and exocuticle at the site of fracture as fixed effects, including an interaction term, and with maximum tensile force sustained prior to fracture as a response variable.
				This model was then compared to models with only cross-sectional area of either endocuticle or exocuticle as the sole fixed effect in the model.
				We then tested the first hypothesis by using likelihood-ratio tests and $R^{2}_{\beta*}$-difference tests between each of the three models.
				
				The hypothesis that 
			
 
			\paragraph*{Model selection and fitting}
			\paragraph*{Estimating phylogenetic signal}
		
			
		\subsection*{Code availability}
			
		\subsection*{Data availability}

%references for methods only
	\begin{thebibliography}{50}
		\section*{References}	
		\bibitem{lamport94}
			Leslie Lamport,
			\textit{\LaTeX: a document preparation system},
			Addison Wesley, Massachusetts,
			2nd edition,
			1994.

	\end{thebibliography}
\end{document}