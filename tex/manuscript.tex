\documentclass[twocolumn, linenumbers, superscriptaddress]{revtex4-1}

\usepackage{graphicx}
\usepackage{lipsum}
\usepackage{blindtext}

\begin{document}
	%abstract - approx. 150 words
	\begin{abstract}
		\blindtext
	\end{abstract}
	
	%title and author block
	{\title{Cuticular composite structure modification in the acorn weevil rostrum}
	
	\date{\today}
	
	\author{M. Andrew Jansen}
		\email[corresponding author, email:~]{majanse1@asu.edu}
		\affiliation{School of Life Sciences, Arizona State University, Tempe, AZ 85287, USA}
	\author{Nikhilesh Chawla}
		\affiliation{School for Engineering of Matter, Energy, and Transport, Arizona State University, Tempe, AZ 85287, USA}
	\author{Nico M. Franz}
		\affiliation{School of Life Sciences, Arizona State University, Tempe, AZ 85287, USA}
		
	\maketitle
	}
	
%introduction - nature materials requires that introduction is given NO SECTION HEADER
<<<<<<< HEAD
	Brief explanation of relevant behavior and biology, prior work on ecology, head shape, cuticle mechanics, and rostral microstructure.
	\lipsum
	
%4 sections of results ideally, nobody really uses subsections, 6 display items max.
	\section*{Microstructure}
		Display 1: Pictures of heads, macrofiber arrangement, and exo-endo ratio at base and apex, across species
		Explain how everything is laid out across the length of the rostrum, emphasizing that this is key to predicting and understanding the mechanical behavior of the snout during bending.
		\lipsum

	\section*{Tensile properties}
		The predicted behavior of the snout is borne out by the data, where UTS is strongly correlated with the cross-sectional area of the endocuticle across species.
		\lipsum
	
	\section*{Fracture mechanics}
		\lipsum
		
	\section*{Fatigue}
=======
	We report novel modifications of the composite microstructure of the the exoskeleton in the snout of acorn weevils.
	These modifications enable the snout to be flexed until straight, remaining within the elastic limits of the material, and without evident alteration of the mechanical properties of the individual components of the cuticle across the structure.
	Thus, the flexibility of the rostrum appears to be derived exclusively through modification of the composite architecture and fiber arrangement in the exoskeleton.
	Support for this hypothesis has come from three lines of evidence: first, examination of the cuticle microstructure across the length of the snout has revealed modifications to the composite structure of the rostrum; second, tensile testing of the structure has demonstrated that the mechanical strength of the cuticle components are consistent along the length of the structure among species; and, third, fatigue testing has shown that a highly curved rostrum is capable of flexing hundreds of thousands of times without damage to the structure.
	We additionally report on the fracture mechanics of the snout, as pertains to both cuticle composite structure and tensile behavior, and consider how modification of the cuticle may reduce the risk of rostral fracture during oviposition.
	To our knowledge, this is the first time such modifications have been reported for enhancing structural elasticity in the insect exoskeleton.
	
%4 sections of results ideally, nobody really uses subsections, 6 display items max.
	\section*{Microstructure}
		\lipsum
	
	\section*{Tensile testing and fracture mechanics}
		\lipsum
		
	\section*{Fatigue testing of \textit{Curculio caryae}}
>>>>>>> 7db52e24b4298f3a7571c3883a716f94dd6524a4
		\lipsum

%conclusion	- with intro and "results" should be no more than 3000 words
	\section*{Conclusion}
		\blindtext[3]

%actual methods after references
	\section*{Methods}
		Methods, including statements of data availability and any associated accession codes and references, are available in the online version of this paper.
	
%50 references (as a guideline)
	\begin{thebibliography}{50}
		\section*{References}	
		\bibitem{lamport94}
			Leslie Lamport,
			\textit{\LaTeX: a document preparation system},
			Addison Wesley, Massachusetts,
			2nd edition,
			1994.

	\end{thebibliography}

%brief and 'without effusive statements'
%thank everyone who helped with data collection, especially jason (tensile testing), sridhar (if I use micro-CT scan data), dave (SEM), charlie (specimens), paige (if I use confocal data), Sal (if I use histology/staining data or confocal data, he helped with specimen clearing).
%Financial support/ funding sources/ grants.
	\begin{acknowledgements}
		\blindtext
	\end{acknowledgements}

%author contributions
%MAJ: did tensile tests, SEM, confocal (if applicable), data analysis.
%MAJ, NC: designed research, micro-CT (if applicable)
%MAJ, NMF: field collection of test specimens, procurement of specimen loans, specimen prep.
%MAJ, NC, NMF: All authors contributed to the writing, review, and revision of the manuscript.
	\section*{Author contributions}
		\blindtext
	
	\section*{Additional information}
		Supplementary information is available in the online version of the paper.
		Reprints and permissions information is available online at www.nature.com/reprints.
		Correspondence and requests for materials should be addressed to M.A.J.
	
	\section*{Competing financial interests}
		The authors declare no competing financial interests.
	
	\newpage

%provide detailed methods here, subsection for each method, statistical analysis, statements of code and data avilability.
	\section*{Methods}
		\subsection*{Method 1}
			\blindtext[3]
			
		\subsection*{Method 2}
			\blindtext[3]
			
		\subsection*{Method 3}
			\blindtext[3]
			
		\subsection*{Statistical analysis}
			\blindtext[3]
			
		\subsection*{Code availability}
			\blindtext[3]
			
		\subsection*{Data availability}
			\blindtext[3]

%references for methods only
	\begin{thebibliography}{50}
		\section*{References}	
		\bibitem{lamport94}
			Leslie Lamport,
			\textit{\LaTeX: a document preparation system},
			Addison Wesley, Massachusetts,
			2nd edition,
			1994.

	\end{thebibliography}
\end{document}