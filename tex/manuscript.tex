\documentclass[twocolumn, linenumbers, superscriptaddress]{revtex4-1}

\usepackage{graphicx}
\usepackage{blindtext}
\usepackage{gensymb}
\usepackage{xcolor}
\usepackage[colorlinks]{hyperref}
	\hypersetup{colorlinks,
	linkcolor={red!50!black},
	citecolor={blue!60!black},
	urlcolor={blue!40!black}
	}
\usepackage{cleveref}
\graphicspath{{figures/}}

\begin{document}
	%abstract - approx. 150 words
	\begin{abstract}
		\blindtext
	\end{abstract}
	
	%title and author block
	{\title{Microstructure-derived strength in the acorn weevil exoskeleton}
	
	\date{\today}
	
	\author{M. Andrew Jansen}
		\email[corresponding author, email:~]{majanse1@asu.edu}
		\affiliation{School of Life Sciences, Arizona State University, Tempe, AZ 85287, USA}
	\author{Jason Williams}
		\affiliation{School for Engineering of Matter, Energy, and Transport, Arizona State University, Tempe, AZ 85287, USA}
	\author{Nikhilesh Chawla}
		\affiliation{School for Engineering of Matter, Energy, and Transport, Arizona State University, Tempe, AZ 85287, USA}
	\author{Nico M. Franz}
		\affiliation{School of Life Sciences, Arizona State University, Tempe, AZ 85287, USA}
		
	\maketitle
	}
	
%introduction - nature materials requires that introduction is given NO SECTION HEADER
%Brief explanation of relevant behavior and biology, prior work on ecology, head shape, cuticle mechanics, and rostral microstructure.

%~650 words here

	We report novel modifications to the composite microstructure of the exoskeleton in the snout of acorn weevils (Coleoptera: Curculionidae) belonging to the genus \textit{Curculio} Linnaeus, 1756.

	As a weevil (snout beetle), members of the genus \textit{Curculio} are typified by the presence of a highly elongate structure on the head, called the rostrum (snout). 
	This structure is a hollow, strongly curved (over 90$\degree$), cylindrical extension of the exoskeleton of the otherwise nearly-spherical head, which bears at its apex the terminal chewing mouthparts. 
	The space inside of the rostrum contains the esophagus, various muscles and tendons used for feeding, and hemolymph that serves as a rough equivalent to blood in insects.
	By contrast, the solid shell of the rostrum is comprised entirely of cuticle, which can be considered a laminate composite consisting of various arrangements of chitin fibers embedded in a protein matrix (see \cref{sec:microstructure}).
	Acorn weevils use this structure to excavate sites for egg-laying (oviposition) and feeding on a variety of fruits, including acorns, Japanese camellia, hazelnuts, pecans, chestnuts, and chinquapins. 
		
	During oviposition, a female engages in a unique "drilling" behavior that causes significant, apparently elastic, deformation of the rostrum.
	The female will insert the snout into an incision made with the mandibles, eating the material as she proceeds, while rotating her head and body around the perimeter of the bore-hole.
	Once the apex of the snout is fully inserted, she will push up and forward with her front legs, forcing the rostrum to bend until it is nearly straight.
	The female will maintain tension on the rostrum in this position, continuing to ingest the substrate and rotate around the bore-hole, while slowly inserting the rostrum further into the excavated channel.
	Once the rostrum is fully inserted, usually up to the eyes, she will pull her snout from the bore-hole and deposit several eggs into the site.
	By maintaining constant tension on the rostrum and rotating around the bore-hole, the female is able to flex the snout into a near-perfectly straight configuration and thereby produce a linear channel into the fruit.
	While this behavior has been observed in many species of \textit{Curculio}, we have lacked a fundamental understanding of how female \textit{Curculio} rostra can withstand the repeated, often extreme bending incurred during the process of oviposition.

	We have found that the composite profile of the rostrum is strongly differentiated from the head capsule and other body parts, with modification of both the relative layer thicknesses and fiber orientation angles of cuticle regions (viz. exocuticle and endocuticle), which we describe in detail below.
	We posit that these modifications enable the snout to be flexed until straight while remaining within the elastic limits of the material, mitigating the risk of structural damage, and without evident alteration of the mechanical properties of the individual components of the cuticle across the structure and between species.
	Thus, the flexibility and tensile strength of the rostrum appear to be derived \emph{exclusively} from modification of the composite architecture of the exoskeleton.
	
	Support for this hypothesis has come from three lines of evidence:
	
	\begin{enumerate}
	\item Examination of the cuticle microstructure across the length of the snout has revealed consistent modification to the composite structure of the rostrum among \textit{Curculio} species.
	\item Tensile testing of the rostrum has demonstrated that the mechanical strength of the cuticle components are consistent along the length of the structure and between species.
	\item Fatigue testing has shown that a highly curved rostrum is capable of flexing hundreds of thousands of times without damage to the structure, and is apparently elastic.
	\end{enumerate}
	
	We additionally describe the fracture mechanics of the snout, as pertains to both cuticle composite structure and tensile behavior, and consider how modification of the cuticle may reduce the risk of rostral fracture during oviposition.
	To our knowledge, this is the first time that a modified composite profile has been reported as a means of enhancing structural elasticity in the insect exoskeleton.

	MIGHT BE GOOD TO SIGNPOST HERE... we discuss each of these lines of evidence in the sections below beginning with a consideration of the impact of microstructure, then we talk about mechanical testing, blah blah. Emphasize that this is key to predicting and understanding the mechanical behavior of the snout during bending.
	
%stick to 600 words for each results section and 500 words for conclusions
	\section{Microstructure of the rostrum}\label{sec:microstructure}
	%~700 words here (1500 total, shorten later)
		Insect cuticle is made up of chitin and numerous uncharacterized proteins, which effectively act as a matrix into which the chitin is embedded \cite{Nikolov2011,Nikolov2010,Vincent2004}.
		The arrangement of the embedded chitin fibers is primarily responsible for the mechanical behavior of the cuticle, although this is modulated by the degree of tanning and water retention. \cite{Klocke2011,Vincent2004}.	
		Accordingly, cuticle is often considered as a fiber-reinforced composite material; however, in beetles, the composite arrangement of the chitin fibers varies by cuticle region.	
	
		Insect cuticle is divisible into three regions (Figs.~INSERT-REFS-HERE), including (1) endocuticle, which is the most compliant and innermost region; and (3) exocuticle, which is the stiffest, hardest, and outermost region.
		Between these lies (2) mesocuticle, which is similar in microstructure to exocuticle, but less sclerotized (tanned), usually acting as a thin transition zone \cite{Klocke2011,Vincent1982,Vincent2004}.
		
		In general, these regions are confluent, and not necessarily sharply defined; for instance, mesocuticle is not always evident, and is sometimes considered part of the exocuticle.
		In beetles, both exocuticle and mesocuticle (when present) are laminate, and have numerous laminae of unidirectional chitin fibers, each layer a single fiber thick (2-4nm), embedded in a proteinaceous matrix.
		These layers are stacked at a more or less constant angle to each other in a helicoidal arrangement, referred to as the Bouligand structure \cite{Blackwell1980,Bouligand1972,Neville1976}. 
		This microstructure allows exocuticle and mesocuticle to exhibit transverse isotropy, or complete isotropy in some cases, despite the strong anisotropy of $\alpha$-chitin.
		
		Conversely, the endocuticle of Coleoptera appears to take advantage of this anisotropy to improve fracture resistance of the exoskeleton.
		Beetles in particular have a highly modified endocuticle comprised of large (5-20$\mu$m OD) unidirectional bundles of chitin, called macrofibers.
		Macrofibers are aligned in layers, as in Figs~INSERT-FIGS-HERE \cite{Kamp2010,Kamp2015}.
		Typically, endocuticle contains several layers of macrofibers, with the adjacent layers forming pairs. 
		The layers \emph{within} each pair are pseudo-orthogonal (i.e., angled approx. $90\degree$ to each other, see Figs.~INSERT-REFS-HERE), while the stacking angle \emph{between} pairs is typically acute, although other configurations have been observed \cite{Hepburn1973,Kamp2010}.
		It is thought that this arrangement of the macrofibers contributes to the toughness of beetle exoskeletons by inhibiting crack formation and propagation between successive plies \cite{Kamp2010,Kamp2015,Hepburn1973}.		
		
		We have found that in \textit{Curculio}, the head capsule (which is similar to the rest of the body) fits the general profile of endocuticle, with an angle of approximately $30\degree$ between successive pseudo-orthogonal plies.
		Additionally, in the head capsule, the thickness of the exocuticle in cross section is nearly equal to that of the endocuticle; compared to the snout, this part of the head is fairly rigid.
		
		By contrast, the region beyond the scrobe (antennal channel) is quite flexible, even in fully desiccated specimens.
		Serial thin sectioning of the snout has demonstrated that the cuticle in this region has a different composite structure (see Fig~INSERT-REFS-HERE) than the head capsule, namely:

		\begin{enumerate}
			\item The exocuticle is reduced to a thin shell, with the endocuticle thickened to offset this reduction.
			\item The endocuticular macrofibers exhibit no rotation between successive pseudo-orthogonal plies, which are all oriented at approx. $\pm45\degree$ to the longitudinal axis of the snout.
		\end{enumerate}
		
		Correspondingly, the portion of the snout between the head capsule and apex of the scrobe exhibits a gradual transition in composite profile along an anterior-posterior gradient.		
		In previous work we identified these modifications to the composite structure of the cuticle, but only within a single species, \textit{C. longinasus} Chittenden, 1927 \cite{Jansen2016, Singh2016}.
		This composite profile has now been uncovered in the rostral apex of six additional, phylogenetically disparate, species (listed below), indicating that this is likely a genus-wide trait.
		
		Notably, we could find no evidence of resilin, as indicated by florescent microscopy, anywhere in the cuticle of the head (including the rostrum).
		We also observed no difference in total cuticle thickness between the head capsule and rostral apex, only differences in relative layer thicknesses.
		Below we demonstrate that that the endocuticle does not vary in tensile strength across the rostrum, making it unlikely that differences in sclerotization or chitin composition within the cuticle are responsible for the mechanical behavior of the rostrum.
		The available evidence therefore suggests that the relative flexibility of the snout is solely derived from the composite profile of its cuticle.
	
	\section{Tensile testing of the rostrum} %shoot for 600 words
		To better characterize the mechanical behavior of the snout as an integrated whole, we performed tensile testing on the snouts of six species in the genus \textit{Curculio}, representing a mixture of closely and distantly related taxa.
		Although the heads were rehydrated by immersion in de-ionized water for 24 hours, we observed comparatively brittle fracture (but see below), in contrast to GIVE EXAMPLES HERE.
		The tensile behavior for the cuticle of these weevils is a result of its microstructure, which lacks pores, etc (give reasons).
		THIS IS ALSO A GOOD SPOT TO SIGNPOST WHAT I FOUND IN GENERAL.
		
		\subsection{Tensile behavior}
			We observed that the maximum force sustained at the site of the break was strongly correlated with the cross-sectional area of the endocuticle, and not the exocuticle.
			Consequently, there was a negative correlation between utlimate tensile strenth of the specimen and the ratio of exocuticle to endocuticle cross-sectional area at the site of fracture, indicating that as the proportion of exocuticle increased, tensile strength decreased.
			In other words, UTS is strongly correlated with the cross-sectional area of the endocuticle across species.
			These associations were found to be statistically significant and independent of species membership, rostrum length, and location on the snout.
			
			These data have three important implications:
			\begin{enumerate}
				\item There is very little variation in the gross elastic behavior of the cuticle across the genus, in agreement with our current understanding of cuticle mechanobiology. 
				\item The endocuticle contributes more to the tensile strength of the rostrum than the exocuticle, possibly because the endocuticle is organized in large bundles of aligned, anisotropic fibers.
				\item Thus, in addition to making the cuticle more flexible, the altered apical composite profile makes it less likely to break, suggesting a possible means by which the system evolved (via negative selection of breakage, maybe say this in conclusions).
			\end{enumerate}
			
			Finally, we observed that Young's modulus of elasticity is higher in specimens with a longer snout.
			This observation was initially quite puzzling, as we had expected that a longer (and typically more strongly curved [insert ecomorph paper ref here]) would need to be more flexible to avoid fracture during oviposition.
			Based on preliminary confocal microscopy data, we speculate that this phenomenon may be the result of a longer scrobe/transition/gradient from the basal profile to the apical profile in longer rostra.
			A decrease in exocuticle thickness along a longer portion of the base might reinforce the snout against buckling; however Young's modulus of the rostrum would be comparatively higher in these species because of the greater volume fraction of exocuticle.
			[OR SAY MORE CONCISELY THAT: based on prelim CLSM data we believe that the snout is reinforced against bucking with thicker exocuticle in the base of these species; as a result, the higher volume fraction of exocuticle in the snout increases young's modulus for such species]
			
		\subsection{Fractography}
			Examination of the fracture surfaces and adjacent cuticle of tensile testing specimens revealed that although fracture was comparatively brittle, the fracture mechanics of the exocuticle and endocuticle differed according to their microstructure, in agreement with previous studies (name some).
			
			In cross-section, the exocuticle consistently presented a smooth, nearly continuous fracture surface, indicative of relatively brittle fracture, likely due to the microarchitecture and resultant transverse isotropy of the Bouligand structure (explain this).
			Endocuticle, on the other hand exhibited severe delamination, ply-splitting, and fiber pulling, consistent with viscoelastic/plastic behavior shown in previous studies and congruent with theoretical consideration of the microstructure of this material (name something here). 
			These patterns indicate that the endocuticle is probably less brittle than the exocuticle, most likely due to the alignment of the $\alpha$-chitin fibers in each macrofiber.
			
			In addition, the exocuticle typically appears to fracture before the endocuticle, with shear-cusp formation evident over unbroken endocuticle.
			We note, however, that the exocuticle of weevils/beetles is anchored to the endocuticle by cross-linking fibers in a transition zone described by Kamp et al. (see refs.).
			The presence of shear-cusps therefore indicates that the fibers of the endocuticle are shearing past each other within each ply along the radial/normal plane (type II shearing???).
			Furthermore, given the delamination observed between plies, the layers of endocuticle are liekly shearing past each other (type III shearing) along the transverse plane.
			We therefore infer that, under tension, the endocuticle tends to deform visco-elastically and plastically along the longitudinal axes of the macrofibers, while the overlying exocuticle exhibits brittle fracture due to shearing between the stretching endocuticle fibers to which it is anchored.
			
			Additionally, the fracture surfaces show a characteristic failure mode, based on the pattern of fiber dislocation in the plies of the endocuticle.
			(copy description from DATA PRESENTATION, for figure 5B).
			
			From this pattern we hypothesize that the exocuticle-rich gular sutures are the most likely site for the initiation of void nucleation and failure of the integrated rostral cuticle in cross section.
			Structure failure would take place as cracks propagate through the endocuticle from these sutures, which penetrate the entire thickness of the laminate.
			We speculate that this could be the reason why the cross-sectional profile of \textit{C. caryae} is flattened ventrallyl
			Flattening this region may reduce tensile-strain across the gular sutures when the snout is bent dorsally, thus reducing the risk of fracture in the elongate, strongly-curved rostrum in this species.
		
	\section{Fatigue testing of \textit{Curculio caryae}} %shot for 600 words
		The shearing motion and accommodation of strain led us to question how the cuticle might accommodate repeated strain, as is seen in the living organism.
		We therefore performed fatigue testing on a female curculio longinasus, which exhibits the most extreme degree of bending in the sis species examined.
			
		~400K cycles, complete elongation, rehydrated, coated in grease to prevent loss of moisture and stiffening of the specimen.
		We observed visco-elastic behavior in the specimen, as indicated by histeresis in the stress-strain relationship during each cycle.
		Fmax decreased logarithmically with cycle number, etc., and the specimen appeared to have deformed plastically during the test.
		We initially believed that this indicated damage to the specimen; however, after cleaning the specimen in a 24 hour wash with ethanol and water, we observed that the specimen returned to its original shape.
		The specimen did not show any evidence of fractures or shear cusps anywhere in the surface of the exocuticle, and, furthermore, the tensile strength of the specimen was consistent with other members of its species.
		Given this surprising result, it appears that the specimen was undamaged by the testing.
		We therefore believe that under normal conditions in life, repeated bending of the snout does exceed the yield/plastic limit of the cuticle, and the bending strain is purely elastic/visco-elastic.
		
		We cannot fully account for these results, but we speculate that the microstructure of the endocuticle is resposible for what we observed.
		The endocuticle is made of aligned $\alpha$-chitin nanofibrils whose crystaline structure is enforced by hydrogen bonds between individual chitin chains along their length.
		The viscoelasticity of the cuticle is thought, in part, to come from slippage between these chains as the hydrogen bonds break and reform in response to shearing between the chitin molecules.
		We believe that repeated strain may have caused such slippage in the endocuticle of the rostral apex during the fatigue test, but without sufficient time for the material to \textit{completely} relax after deformation, the specimen would slowly accumulate strain and consequently deform visco-elastically/visco-plastically.
		After 24 hours soaking in ethanol and water, the hydrogen bonds would relax sufficiently to allow the specimen to return to its original configuration, dissipating the accumulated strain without any damage to the specimen.

	\section{Conclusions} %shoot for 500 words

%actual methods after references
	\section{Methods}
		Methods, including statements of data availability and any associated accession codes and references, are available in the online version of this paper.
	
%50 references (as a guideline, in order of appearance in text)
	\begin{thebibliography}{50}
		\section*{References}	
			\bibitem{Klocke2011}
				Klocke, D. \& Schmitz, H.
				Water as a major modulator of the mechanical properties of insect cuticle.
				\textit{Acta Biomater.}
				\textbf{7,}
				2935--2942
				(2011).
				
			\bibitem{Vincent1982}
				Vincent, J. F. V.
				\textit{Structural biomaterials}
				(Halsted Press,
				New York,
				1982).
				
			\bibitem{Vincent2004}
				Vincent, J. F. V. \& Wegst, U. G. K.
				Design and mechanical properties of insect cuticle.				
				\textit{Arthropod Struct. Dev.}
				\textbf{33:3,}
				187--199,
				(2004).
				
			\bibitem{Nikolov2011}
				Nikolov, S. et al.
				Robustness and optimal use of design principles of arthropod exoskeletons studied by ab initio-based multiscale simulations.
				\textit{J. Mech. Behav. Biomed. Mater.}
				\textbf{4:2,}
				129--145,
				(2011).
				
			\bibitem{Nikolov2010}
				Nikolov, S. et al.
				Revealing the design principles of high-performance biological composites using Ab initio and multiscale simulations: The example of lobster cuticle.
				\textit{Adv. Mater.}
				\textbf{22:4,}
				519--526,
				(2010).
				
			\bibitem{Blackwell1980}
				Blackwell, J. \& Weih, M.
				Structure of chitin-protein complexes: ovipositor of the ichneumon fly \textit{Megarhyssa}.
				\textit{J. Mol. Biol.}
				\textbf{137:1,}
				49--60,
				(1980).

			\bibitem{Bouligand1972}
				Bouligand, Y.
				Twisted fibrous arrangements in biological materials and cholesteric mesophases.
				\textit{Tissue Cell}
				\textbf{4:2,}
				189--217,
				(1972).
				
			\bibitem{Neville1976}
				Neville, A. C., Parry, D. A. \& Woodhead-Galloway, J.
				The chitin crystallite in arthropod cuticle.
				\textit{J. Cell Sci.}
				\textbf{21:1,}
				73--82,
				(1976).
				
			\bibitem{Cheng2009}
				Cheng, L., Wang, L., \& Karlsson, A. M.
				Mechanics-based analysis of selected features of the exoskeletal microstructure of \textit{Popillia japonica}.
				\textit{Mater. Res.}
				\textbf{24:11,}
				3253--3267,
				(2009).
				
			\bibitem{Hepburn1973}
				Hepburn, H. R. \& Ball, A.
				On the structure and mechanical properties of beetle shells.
				\textit{J. Mater. Sci.}
				\textbf{8:5,}
				618--623,
				(1973).
				
			\bibitem{Kamp2010}
				van de Kamp, T. \& Greven, H.
				On the architecture of beetle elytra.
				\textit{Entomol. Heute}
				\textbf{22,}
				191--204,
				(2010).
			
			\bibitem{Kamp2015}
				van de Kamp, T., Riedel, A. \& Greven, H.
				Micromorphology of the elytral cuticle of beetles, with an emphasis on weevils (Coleoptera : Curculionoidea)
				\textit{Arthropod Struct. Dev.}
				\textbf{45:1,}
				14--22,
				(2015).
			
	\end{thebibliography}

%brief and 'without effusive statements'
%thank everyone who helped with data collection, dave (SEM), charlie (specimens), paige (if I use confocal data), Sal (if I use histology/staining data or confocal data, he helped with specimen clearing).
%Financial support/ funding sources/ grants.
	\begin{acknowledgements}

	\end{acknowledgements}

%author contributions
%MAJ: did tensile tests, SEM, confocal (if applicable), data analysis.
%MAJ, NC: designed research, micro-CT (if applicable)
%MAJ, NMF: field collection of test specimens, procurement of specimen loans, specimen prep.
%MAJ, NC, NMF: All authors contributed to the writing, review, and revision of the manuscript.
	\section*{Author contributions}
		\begin{description}
		\item[Andrew Jansen] Conducted sectioning and staining, microscopy and imaging, tensile and fatigue testing, statistical analysis, and participated in manuscript preparation.
		\item[Jason Williams] Conducted tensile and fatigue testing, participated in manuscript preparation.
		\item[Nikhilesh Chawla] Facilitated microscopy, tensile and fatigue testing, and participated in manuscript preparation.
		\item[Nico Franz] Facilitated specimen acquisition and imaging, participated in manuscript preparation.
		\end{description} 
	
	\section*{Additional information}
		Supplementary information is available in the online version of the paper.
		Reprints and permissions information is available online at www.nature.com/reprints.
		Correspondence and requests for materials should be addressed to M.A.J.
	
	\section*{Competing financial interests}
		The authors declare no competing financial interests.
	
	\newpage

%provide detailed methods here, subsection for each method, statistical analysis, statements of code and data avilability.
	\section*{Methods}
		\subsection*{Histological sectioning}
			
		\subsection*{Tensile and fatigue testing}

		\subsection*{Specimen imaging and microscopy}
			
		\subsection*{Statistical analysis}
			\paragraph*{General approach}
				To explore the relationships between the composite structure and mechanical properties of the cuticle, we fit phylogenetic linear mixed-effects models to the data using maximum likelihood estimation.
				In order to control for phylogenetic non-independence in the data, we included the species of each specimen as a random effect in all models.
				We also allowed for correlation in the error term of the models, as specified by a variance-covariance matrix generated from a Brownian motion model of trait evolution along the phylogeny.
				Response variables and covariates were natural-log transformed, as needed, to ensure model residuals were normally distributed and homoscedastic.
				In all models, we tested whether the inclusion of phylogenetic correlation in the model error produced significantly better model fit, using a likelihood-ratio test and $R^{2}_{\sigma}$-difference test between the fully-specified model and a model lacking the phylogenetic effect.
			\paragraph*{Hypothesis testing}
				The following three hypotheses were tested using PGLMMs fitted using ML estimation:
				\begin{enumerate}
				\item The maximum sustained tensile force is proportional to the cross-sectional area of the endocuticle, and \emph{not} that of the exocuticle.
				\item The ultimate tensile strength of the samples is inversely proportional to the ratio of exocuticle to endocuticle at the location of fracture.
				\item Young's modulus of the samples is proportional to the length of the snout.
				\end{enumerate}
				
				We fitted a fully-specified model with the cross-sectional area of endocuticle and exocuticle at the site of fracture as fixed effects, including an interaction term, and with maximum tensile force sustained prior to fracture as a response variable.
				This model was then compared to models with only cross-sectional area of either endocuticle or exocuticle as the sole fixed effect in the model.
				We then tested the first hypothesis by using likelihood-ratio tests and $R^{2}_{\beta*}$-difference tests between each of the three models.
				
				The hypothesis that 
			
 
			\paragraph*{Model selection and fitting}
			\paragraph*{Estimating phylogenetic signal}
		
			
		\subsection*{Code availability}
			
		\subsection*{Data availability}

%references for methods only
	\begin{thebibliography}{50}
		\section*{References}	
		\bibitem{lamport94}
			Leslie Lamport,
			\textit{\LaTeX: a document preparation system},
			Addison Wesley, Massachusetts,
			2nd edition,
			1994.

	\end{thebibliography}
\end{document}