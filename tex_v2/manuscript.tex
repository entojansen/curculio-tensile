\documentclass[twocolumn, linenumbers, superscriptaddress]{revtex4-1}

\usepackage{amsmath}
\usepackage{graphicx}
%TODO remove package: blindtext
\usepackage{blindtext}
\usepackage{gensymb}
\usepackage{xcolor}
\usepackage[colorlinks]{hyperref}
	\hypersetup{colorlinks,
	linkcolor={red!50!black},
	citecolor={blue!60!black},
	urlcolor={blue!40!black}
	}
\usepackage{cleveref}
\graphicspath{{figures/}}

%TODO NOTES! REMOVE FROM COMPLETE DRAFT-----------------------------------------
\usepackage{xargs}
\usepackage[colorinlistoftodos,prependcaption,textsize=tiny]{todonotes}
\newcommandx{\unsure}[2][1=]{\todo[linecolor=red,backgroundcolor=red!25,bordercolor=red,#1]{#2}}
\newcommandx{\change}[2][1=]{\todo[linecolor=blue,backgroundcolor=blue!25,bordercolor=blue,#1]{#2}}
\newcommandx{\info}[2][1=]{\todo[linecolor=green,backgroundcolor=green!25,bordercolor=green,#1]{#2}}
\newcommandx{\improvement}[2][1=]{\todo[linecolor=purple,backgroundcolor=purple!25,bordercolor=purple,#1]{#2}}
\newcommandx{\noshow}[2][1=]{\todo[disable,#1]{#2}}
%-------------------------------------------------------------------------------

\begin{document}
	%abstract - approx. 150 words
	\begin{abstract}
		\blindtext
	\end{abstract}
	
	%title and author block
	{\title{Exoskeletal microstructure and tensile behavior of the acorn weevil rostrum}
	%CONSIDER CHANGING THE TITLE AFTER WRITING MANUSCRIPT
	\date{\today}
	
	\author{M. Andrew Jansen}
		\email[corresponding author, email:~]{majanse1@asu.edu}
		\affiliation{School of Life Sciences, Arizona State University, Tempe, AZ 85287, USA}
	\author{Jason Williams}
		\affiliation{School for Engineering of Matter, Energy, and Transport, Arizona State University, Tempe, AZ 85287, USA}
	\author{Nikhilesh Chawla}
		\affiliation{School for Engineering of Matter, Energy, and Transport, Arizona State University, Tempe, AZ 85287, USA}
	\author{Nico M. Franz}
		\affiliation{School of Life Sciences, Arizona State University, Tempe, AZ 85287, USA}
		
	\maketitle
	}
	
{%manuscript outline:
%3000 word limit, 150 word abstract, 500 word intro, 250 word fig caps

%standard intro, talk about behavior, rostrum bending, and cuticle
%weevil cuticle is known to be stronger in compression than in tension
%in Curculio this means that the greatest risk of breakage would come from tension during bending of the snout
%curculio appear to have mitigated the risk of breakage by making the rostral apex flexible
%prior work suggests that this flexibility comes from a modified composite profile in the cuticle of the rostral apex
%different parts of the head thus appear to have differently specialized cuticle to deal with extreme bending
%only observed in 1 species, never any comparison across species/in genus, fractographic analysis never attempted to explain how snout is most likely to break/ how the cuticle fails (ie what failure modes are the weevils fighting against/trying to mitigate)

%we therefore set out to characterize the modifications to the cuticle of the rostral apex that presumably allow bending, using six species representing a mix of phylogenetically related and disparate species.
%because material rigidity and toughness are often inversely proportional, we also investigate how the composite structure of the cuticle contributes to the tensile strength of the snout, whether the flexible regions exhibit higher tensile strenth, and conduct fractographic analysis on tensile testing specimens to determine the failure modes of the rostrum/cuticle under tension.
%We additionally present the results of a load cycling test for a representative specimen of a species with a highly elongate and curved snout to explore the possibility that bending of the rostrum results in elastic deformation of the cuticle.

%This study demonstrates: (1) there are 2 microstructural modifications to the composite profile of the cuticle of the rostral apex that would enhance cuticular flexibility, present in all examined species, and likely across the genus (2) thickness of the encocuticle is directly proportional to the maximum tensile force sustained in tension, to the exclusion of the exocuticle, (3) the tensile strenth of the snout varies across the length of the head and is inversely proportional to exo/endocuticle ratio, such that the more flexible areas are also stronger under tension, (4) young's modulus is higher for longer snouts, reflecting increased reinforcement at the base of the snout, likely to prevent bucking fracture at the base, (5) because of the modifications to the cuticle of the rostral apex (which result in all of these changes to the mechanical properties of the snout), bending appears to be viscoelastic/elastic, with a relatively low risk of breakage during the lifetime of the beetle, (6) when the rostrum does fail, crack initiation appears to take place in the exocuticle-rich gular sutures, which would be under tension during the drilling behavior in the live insect.

%in conclusion, the cuticle of the rostral apex is modified to increase flexibility, allowing for nearly elastic bending, while simultaneously increasing tensile strength, thus resisting the predominant fracture modes observed in unaltered cuticle.
%to our knowledge, this is the first time that modifications to the composite profile of the cuticle have been observed to optimized mechanical behavior in insects, or something like that.

%this has important implications for understanding the evolution of this highly exaggerated structure, ie suggesting negative selection of breakage instead of positive selection of flexibility. possibly a single ancestral shift for each modification, opened a large region of morphospace for exploration, resulting in present day diversity of the genus.

%-------------------------------------------------------------------------------
%introduction - nature materials requires that introduction is given NO SECTION HEADER, 500 words
}	
\section*{Introduction}
%Intro outline - 500 word maximum
%	Paragraph 1 - Cuticle of beetles
%	Paragraph 2 - Weevils and behavior
%	Paragraph 3 - modified cuticle in curculio
%	Paragraph 4 - study objectives
%	Paragraph 5 - short segue/ signposting
	{The exoskeleton of Coleoptera (beetles) is a hierarchically-structured fibrous composite typified by variously arranged $\alpha$-chitin (N-acetylglucosamine) nanofibrils embedded in a heterogeneous protein matrix.
	Although $\alpha$-chitin is brittle and strongly anisotropic\unsure{give some numbers here}, beetle cuticle is simultaneously rigid and tough due to its uniquely layered microstructure\unsure{give more numbers here}.
	Beetle cuticle is divided into two structurally and mechanically distinct regions, the (outer) exocuticle and (inner) endocuticle.
	The exocuticle is characterized by a transversely-isotropic helicoidal structure that is comparatively rigid (though brittle) and has been the subject of many prior studies\unsure{numbers and refs}.
	By contrast, the endocuticle of beetles is comprised of large, aligned bundles (macrofibers) of chitin that are strongly anisotropic\unsure{numbers again}.
	Arranged in unidirectional laminae, the macrofibers improve cuticle toughness by inhibiting crack formation and propagation between successive plies \cite{Kamp2010,Kamp2015,Hepburn1973}.
	
	In general, impact-prone areas and exaggerated structures, such as horns and legs exhibit cuticle organization that resists deformation and fracture.
	Acorn weevils in the genus \textit{Curculio} are typified by an elongate structure of the head, called the rostrum (snout), which instead exhibits unusual distal flexibility.
	The rostrum is a hollow, strongly curved (over 90$\degree$ in some species), cylindrical, exoskeletal extension of the otherwise nearly-spherical head, which bears at its apex the terminal chewing mouthparts.
	Despite being composed of the same material as other rigid body parts, the snout can be repeatedly bent without evident damage.
	This structure is used by the female to feed and excavate sites for egg-laying (oviposition); the latter process causes significant, apparently elastic, deformation of the rostrum\improvement{as illustrated in figure 1c}.
	By maintaining constant tension on the snout and rotating around the bore-hole, females are able to flex the rostrum into a near-perfectly straight configuration and thereby produce a linear channel into the fruit.
	
	While this behavior has been observed in many species of \textit{Curculio}, it was unclear how the rostrum of female acorn weevils can withstand the repeated, often extreme bending incurred during the process of egg-chamber excavation.
	In this study we characterize the composite profile of the rostral cuticle to account for the observed flexibility of the snout.
	We show that the relative layer thicknesses and fiber orientation angles of the exocuticle and endocuticle of the rostrum are strongly differentiated from the head capsule and other body parts, and we predict the effect of these differences using Classical Laminate Theory (CLT).
	Because recent studies have shown that the yield strength of the weevil rostrum exoskeleton is lower in tension than compression, we perform a comparative analysis of the ultimate tensile strength of the rostrum across species and snout morphotypes; we also report the results of displacement-controlled load cycling of the snout in a species with strongly curved morphology.
	We relate an observed increase in the volume fraction of endocuticle in the rostrum to higher tensile strength at the rostral apex in all tested species, and find that a strongly curved rostrum can be flexed repeatedly without harm to the structure.
	
	We additionally describe the fracture mechanics of the snout, as pertains to both cuticle composite structure and tensile behavior, and consider how modification of the cuticle may reduce the risk of rostral fracture during oviposition.
	Based on our findings we posit that the composite profile of the rostral apex enables the snout to be flexed until straight while remaining within the elastic limits of the material, mitigating the risk of structural damage, and without evident alteration of the mechanical properties of the individual components of the cuticle across the structure and between species.
	Thus, the flexibility and tensile strength of the rostrum appear to be derived \emph{exclusively} from modification of the composite architecture of the exoskeleton.
	To our knowledge, this is the first time that a modified composite profile has been reported as a means of enhancing structural elasticity in the insect exoskeleton.\info{630 words, reduce to 600 or less in revisions, structure of argument/narrative is good}
}	
	\section{Microstructure of the \textit{Curculio} rostrum}
		In arthropods (including beetles), the exocuticle is comprised of numerous unidirectional laminae of chitin fibers; each layer is the thickness of a single fiber (2-4nm) embedded in a proteinaceous matrix.
		These layers are stacked at a more or less constant angle to each other, thus forming a quasi-isotropic laminate referred to as the Bouligand structure \cite{Blackwell1980,Bouligand1972,Neville1976}. 
		This layout effectively produces a transversely isotropic composite, mitigating the strong anisotropy of $\alpha$-chitin to yield a versatile building material for the exoskeleton.
		
		Conversely, the endocuticle of Coleoptera retains the anisotropy of $\alpha$-chitin to improve fracture resistance of the cuticle.
		Beetle endocuticle is unique among arthropods and is comprised of large (5-20$\mu$m OD) unidirectional bundles of chitin, called macrofibers.
		Chitin macrofibers are orthotropic \improvement{insert E1, E2, E3 here} and arranged in unidirectional plies, as in Figs. \info{~INSERT-FIGS-HERE} \cite{Kamp2010,Kamp2015}.
		Typically, adjacent macrofiber plies are paired and pseudo-orthogonal (i.e., angled approx. $90\degree$ to each other, see Figs.\info{~INSERT-REFS-HERE}), with a constant stacking angle \emph{between} pairs \info{what kind of laminate is this?}, although other configurations have been observed \cite{Hepburn1973,Kamp2010}.
		This geometric sequence of the macrofiber laminae yields an approximately transversely isotropic composite, similar to the Bouligand structure.
		Notably, the resulting laminate is less rigid than the exocuticle, but exhibits greater toughness because the pseudo-orthogonal plies effectively inhibit crack formation and propagation between successive layers \cite{Kamp2010,Kamp2015,Hepburn1973}.
		
		Serial thin sectioning and SEM of fractured \textit{Curculio} specimens has revealed that endocuticle in the head capsule fits this general profile, with an angle of approximately $30\degree$ between successive pairs of pseudo-orthogonal plies.
		Additionally, in the head capsule, the thickness of the exocuticle in cross section is nearly equal to that of the endocuticle\info{thickness measurements here}.
		However, we have also found that the cuticle composite lay-up of the rostral apex differs from that of the head capsule (see Fig.\info{~INSERT-REFS-HER}E) in two key characteristics:

		\begin{enumerate}
			\item The exocuticle is reduced to a thin shell\info{measurement here}, with the endocuticle thickened \info{measurement here} to offset this reduction to maintain a constant cuticle thickness in the head.
			\item The endocuticular macrofibers exhibit no rotation between successive pseudo-orthogonal plies, which are all oriented at approximately $\pm45\degree$ to the longitudinal axis of the snout (i.e., an antisymmetric $[\pm45\degree]$ angle-ply laminate).
		\end{enumerate}
		
		In previous work we identified these modifications to the composite structure of the cuticle within a single species, \textit{C. longinasus} Chittenden, 1927 \cite{Jansen2016, Singh2016}.
		This composite profile has now been uncovered in the rostral apex of six additional, phylogenetically disparate, species (listed below), indicating that this is likely a genus-wide trait.
		We note that in all species, the portion of the snout between the head capsule and apex of the scrobe exhibits a gradual transition in composite profile along an anterior-posterior gradient.
		
		To investigate the effect of these cuticle modifications, we estimated the engineering and flexural elastic moduli of the cuticle in both the rostral apex and head capsule using Classical Laminate Plate Theory (CLPT), as detailed in our methods.
		We previously derived the effective elastic constants of the cuticle regions of \textit{C. longinasus}, which we used here to construct constitutive equations for the entire cuticle of that species.
		The cuticle of the head capsule has moduli\info{insert numbers here}.
		The cuticle of the rostral apex has moduli\info{insert numbers here}.
		
		Interestingly, each of these modifications contributes to the observed flexibility of the snout.
		
		\info{some explanation in source code}
		\noshow{Because the exocuticle is generally more rigid (and brittle) than the endocuticle due to its transverse isotropy and microstructure, a reduction in the relative thickness of this region would likely have the effect of increasing the flexibility of the resulting composite.
		Additionally, aligning the layers of endocuticle as a 45 deg cross ply would likely reduce the bending moment of the structure.
		The combined effect of these modifications is that the apex of the rostrum is able to bend until completely straight without fracture.}
		
		Below we demonstrate that the endocuticle does not vary in tensile strength across the rostrum or between species, making it unlikely that differences in sclerotization or chitin composition within the cuticle are responsible for the mechanical behavior of the rostrum.
		The available evidence therefore suggests that the relative flexibility of the snout is primarily derived from the composite profile of its cuticle.
		
	\section{Force-controlled loading to fracture}
		To better characterize the mechanical behavior of the snout as an integrated whole, we performed tensile testing on the snouts of six species in the genus \textit{Curculio}, representing a mixture of closely and distantly related taxa.
		Although the heads were rehydrated by immersion in de-ionized water for 24 hours, we observed comparatively brittle fracture (but see below), in contrast to GIVE EXAMPLES HERE.
		The tensile behavior for the cuticle of these weevils is a result of its microstructure, which lacks pores, etc (give reasons).
		THIS IS ALSO A GOOD SPOT TO SIGNPOST WHAT I FOUND IN GENERAL.
		
		We observed that the maximum force sustained at the site of the break was strongly correlated with the cross-sectional area of the endocuticle, and not the exocuticle.
		Consequently, there was a negative correlation between utlimate tensile strenth of the specimen and the ratio of exocuticle to endocuticle cross-sectional area at the site of fracture, indicating that as the proportion of exocuticle increased, tensile strength decreased.
		In other words, UTS is strongly correlated with the cross-sectional area of the endocuticle across species.
		These associations were found to be statistically significant and independent of species membership, rostrum length, and location on the snout.
		
		These data have three important implications:
		\begin{enumerate}
			\item There is very little variation in the gross elastic behavior of the cuticle across the genus, in agreement with our current understanding of cuticle mechanobiology. (singh 2016, jansen 2016)
			\item The endocuticle contributes more to the tensile strength of the rostrum than the exocuticle, possibly because the endocuticle is organized in large bundles of aligned, anisotropic fibers.
			\item Thus, in addition to making the cuticle more flexible, the altered apical composite profile makes it less likely to break, suggesting a possible means by which the system evolved (via negative selection of breakage, maybe say this in conclusions).
		\end{enumerate}
		
		Finally, we observed that the elastic modulus is higher in specimens with a longer snout.
		This observation was initially quite puzzling, as we had expected that a longer (and typically more strongly curved [insert ecomorph paper ref here]) would need to be more flexible to avoid fracture during oviposition.
		Based on preliminary confocal microscopy data, we speculate that this phenomenon may be the result of a longer scrobe/transition/gradient from the basal profile to the apical profile in longer rostra.
		A decrease in exocuticle thickness along a longer portion of the base might reinforce the snout against buckling; however Young's modulus of the rostrum would be comparatively higher in these species because of the greater volume fraction of exocuticle.
		[OR SAY MORE CONCISELY THAT: based on prelim CLSM data we believe that the snout is reinforced against bucking with thicker exocuticle in the base of these species; as a result, the higher volume fraction of exocuticle in the snout increases young's modulus for such species]
	
	\section{Load cycling of \textit{Curculio caryae}}
		The shearing motion and accommodation of strain led us to question how the cuticle might accommodate repeated strain, as is seen in the living organism.
		We therefore performed fatigue testing on a female curculio longinasus, which exhibits the most extreme degree of bending in the sis species examined.
			
		~400K cycles, complete elongation, rehydrated, coated in grease to prevent loss of moisture and stiffening of the specimen.
		We observed visco-elastic behavior in the specimen, as indicated by histeresis in the stress-strain relationship during each cycle.
		Fmax decreased logarithmically with cycle number, etc., and the specimen appeared to have deformed plastically during the test.
		We initially believed that this indicated damage to the specimen; however, after cleaning the specimen in a 24 hour wash with ethanol and water, we observed that the specimen returned to its original shape.
		The specimen did not show any evidence of fractures or shear cusps anywhere in the surface of the exocuticle, and, furthermore, the tensile strength of the specimen was consistent with other members of its species.
		Given this surprising result, it appears that the specimen was undamaged by the testing.
		We therefore believe that under normal conditions in life, repeated bending of the snout does exceed the yield/plastic limit of the cuticle, and the bending strain is purely elastic/visco-elastic.
		
		We cannot fully account for these results, but we speculate that the microstructure of the endocuticle is resposible for what we observed.
		The endocuticle is made of aligned $\alpha$-chitin nanofibrils whose crystaline structure is enforced by hydrogen bonds between individual chitin chains along their length.
		The viscoelasticity of the cuticle is thought, in part, to come from slippage between these chains as the hydrogen bonds break and reform in response to shearing between the chitin molecules.
		We believe that repeated strain may have caused such slippage in the endocuticle of the rostral apex during the fatigue test, but without sufficient time for the material to \textit{completely} relax after deformation, the specimen would slowly accumulate strain and consequently deform visco-elastically/visco-plastically.
		After 24 hours soaking in ethanol and water, the hydrogen bonds would relax sufficiently to allow the specimen to return to its original configuration, dissipating the accumulated strain without any damage to the specimen.
			
	\section{Fractography of test specimens}
		Examination of the fracture surfaces and adjacent cuticle of tensile testing specimens revealed that although fracture was comparatively brittle, the fracture mechanics of the exocuticle and endocuticle differed according to their microstructure, in agreement with previous studies (name some).
			
		In cross-section, the exocuticle consistently presented a smooth, nearly continuous fracture surface, indicative of relatively brittle fracture, likely due to the microarchitecture and resultant transverse isotropy of the Bouligand structure (explain this).
		Endocuticle, on the other hand exhibited severe delamination, ply-splitting, and fiber pulling, consistent with viscoelastic/plastic behavior shown in previous studies and congruent with theoretical consideration of the microstructure of this material (name something here). 
		These patterns indicate that the endocuticle is probably less brittle than the exocuticle, most likely due to the alignment of the $\alpha$-chitin fibers in each macrofiber.
		
		In addition, the exocuticle typically appears to fracture before the endocuticle, with shear-cusp formation evident over unbroken endocuticle.
		We note, however, that the exocuticle of weevils/beetles is anchored to the endocuticle by cross-linking fibers in a transition zone described by Kamp et al. (see refs.).
		The presence of shear-cusps therefore indicates that the fibers of the endocuticle are shearing past each other within each ply along the radial/normal plane (type II shearing???, extension shear-coupling).
		Furthermore, given the delamination observed between plies, the layers of endocuticle are liekly shearing past each other (type III shearing) along the transverse plane.
		We therefore infer that, under tension, the endocuticle tends to deform visco-elastically and plastically along the longitudinal axes of the macrofibers, while the overlying exocuticle exhibits brittle fracture due to shearing between the stretching endocuticle fibers to which it is anchored.
		
		Additionally, the fracture surfaces show a characteristic failure mode, based on the pattern of fiber dislocation in the plies of the endocuticle.
		(copy description from DATA PRESENTATION, for figure 5B).
		
		From this pattern we hypothesize that the exocuticle-rich gular sutures are the most likely site for the initiation of void nucleation and failure of the integrated rostral cuticle in cross section.
		Structural failure would take place as cracks propagate through the endocuticle from these sutures, which penetrate the entire thickness of the laminate.
		We speculate that this could be the reason why the cross-sectional profile of \textit{C. caryae} is flattened ventrally.
		Flattening this region may reduce tensile-strain across the gular sutures when the snout is bent dorsally, thus reducing the risk of fracture in the elongate, strongly-curved rostrum in this species.

	\section{Conclusions}
	%1-2 'short paragraphs'
		

	\section{Methods}
	%actual methods after references
		Methods, including statements of data availability and any associated accession codes and references, are available in the online version of this paper.
	
	\begin{thebibliography}{50}
	%50 references (as a guideline, in order of appearance in text)
		\section*{References}	
			\bibitem{Klocke2011}
				Klocke, D. \& Schmitz, H.
				Water as a major modulator of the mechanical properties of insect cuticle.
				\textit{Acta Biomater.}
				\textbf{7,}
				2935--2942
				(2011).
				
			\bibitem{Vincent1982}
				Vincent, J. F. V.
				\textit{Structural biomaterials}
				(Halsted Press,
				New York,
				1982).
				
			\bibitem{Vincent2004}
				Vincent, J. F. V. \& Wegst, U. G. K.
				Design and mechanical properties of insect cuticle.				
				\textit{Arthropod Struct. Dev.}
				\textbf{33:3,}
				187--199,
				(2004).
				
			\bibitem{Nikolov2011}
				Nikolov, S. et al.
				Robustness and optimal use of design principles of arthropod exoskeletons studied by ab initio-based multiscale simulations.
				\textit{J. Mech. Behav. Biomed. Mater.}
				\textbf{4:2,}
				129--145,
				(2011).
				
			\bibitem{Nikolov2010}
				Nikolov, S. et al.
				Revealing the design principles of high-performance biological composites using Ab initio and multiscale simulations: The example of lobster cuticle.
				\textit{Adv. Mater.}
				\textbf{22:4,}
				519--526,
				(2010).
				
			\bibitem{Blackwell1980}
				Blackwell, J. \& Weih, M.
				Structure of chitin-protein complexes: ovipositor of the ichneumon fly \textit{Megarhyssa}.
				\textit{J. Mol. Biol.}
				\textbf{137:1,}
				49--60,
				(1980).

			\bibitem{Bouligand1972}
				Bouligand, Y.
				Twisted fibrous arrangements in biological materials and cholesteric mesophases.
				\textit{Tissue Cell}
				\textbf{4:2,}
				189--217,
				(1972).
				
			\bibitem{Neville1976}
				Neville, A. C., Parry, D. A. \& Woodhead-Galloway, J.
				The chitin crystallite in arthropod cuticle.
				\textit{J. Cell Sci.}
				\textbf{21:1,}
				73--82,
				(1976).
				
			\bibitem{Cheng2009}
				Cheng, L., Wang, L., \& Karlsson, A. M.
				Mechanics-based analysis of selected features of the exoskeletal microstructure of \textit{Popillia japonica}.
				\textit{Mater. Res.}
				\textbf{24:11,}
				3253--3267,
				(2009).
				
			\bibitem{Hepburn1973}
				Hepburn, H. R. \& Ball, A.
				On the structure and mechanical properties of beetle shells.
				\textit{J. Mater. Sci.}
				\textbf{8:5,}
				618--623,
				(1973).
				
			\bibitem{Kamp2010}
				van de Kamp, T. \& Greven, H.
				On the architecture of beetle elytra.
				\textit{Entomol. Heute}
				\textbf{22,}
				191--204,
				(2010).
			
			\bibitem{Kamp2015}
				van de Kamp, T., Riedel, A. \& Greven, H.
				Micromorphology of the elytral cuticle of beetles, with an emphasis on weevils (Coleoptera : Curculionoidea)
				\textit{Arthropod Struct. Dev.}
				\textbf{45:1,}
				14--22,
				(2015).
			
	\end{thebibliography}

%brief and 'without effusive statements'
%thank everyone who helped with data collection, dave (SEM), charlie (specimens), paige (if I use confocal data), Sal (if I use histology/staining data or confocal data, he helped with specimen clearing).
%Financial support/ funding sources/ grants.
	\begin{acknowledgements}

	\end{acknowledgements}

%author contributions
%MAJ: did tensile tests, SEM, confocal (if applicable), data analysis.
%MAJ, NC: designed research, micro-CT (if applicable)
%MAJ, NMF: field collection of test specimens, procurement of specimen loans, specimen prep.
%MAJ, NC, NMF: All authors contributed to the writing, review, and revision of the manuscript.
	\section*{Author contributions}
		\begin{description}
		\item[Andrew Jansen] Conducted sectioning and staining, microscopy and imaging, tensile and fatigue testing, statistical analysis, and participated in manuscript preparation.
		\item[Jason Williams] Conducted tensile and fatigue testing, participated in manuscript preparation.
		\item[Nikhilesh Chawla] Facilitated microscopy, tensile and fatigue testing, and participated in manuscript preparation.
		\item[Nico Franz] Facilitated specimen acquisition and imaging, participated in manuscript preparation.
		\end{description} 
	
	\section*{Additional information}
		Supplementary information is available in the online version of the paper.
		Reprints and permissions information is available online at www.nature.com/reprints.
		Correspondence and requests for materials should be addressed to M.A.J.
	
	\section*{Competing financial interests}
		The authors declare no competing financial interests.
	
	\newpage

%provide detailed methods here, subsection for each method, statistical analysis, statements of code and data avilability.
	\section*{Methods}
		\subsection*{Histological sectioning}
			
		\subsection*{Tensile and fatigue testing}

		\subsection*{Specimen imaging and microscopy}
		
		\subsection*{Constitutive modeling of the cuticle}
			\paragraph*{General Approach}
			\paragraph*{Cuticle profile of model}
				The cuticle of \textit{C. longinasus} is 50 microns thick; we use \textit{C. longinasus} because we have constitutive models for the cuticle regions of this species.
				We assumed equal layer thicknesses in the endocuticle of the basal cuticle, and equal thicknesses of exocuticle and endocuticle.
				In the apical cuticle we assumed the exocuticle and upper 8 layers of the endocuticle were each 5 microns thick, with 4 thinner layers of endocuticle, each 1.5 microns thick.
				The angles of the endocuticle layers for both types of cuticle are as described above.
			\paragraph*{Permutations}
				We additionally calculated models for two hypothetical hybrid cuticles: one model has the layer thicknesses of the apex, but fiber orientations of the base, while the second has the fiber orientations of the apex, but the layer thicknesses of the base.
			\paragraph*{Classical Laminate Plate Theory}
				We begin by calculating the 2D reduced stiffness matrix for each part of the cuticle.
				For orthotropic materials with the principal axes parallel to the ply edges, the reduced stiffness matrix is defined as follows:
				
				\begin{equation}
				[Q] =
					\begin{bmatrix}
						Q_{11} & Q_{12} & 0 \\
						Q_{21} & Q_{22} & 0 \\
						0 & 0 & Q_{66}
					\end{bmatrix}\,,
				\end{equation}
				
				and where:
				
				\begin{equation}
				\begin{aligned}
					Q_{11} & = \frac{E_{1}}{1 - \nu_{12}\nu_{21}}\,, \\
					Q_{12} & = \frac{E_{1}\nu_{21}}{1 - \nu_{12}\nu_{21}} = Q_{21}\,, \\
					Q_{21} & = \frac{E_{2}\nu_{12}}{1 - \nu_{12}\nu_{21}} = Q_{12}\,, \\
					Q_{22} & = \frac{E_{2}}{1 - \nu_{12}\nu_{21}}\,, \\
					Q_{66} & = G_{12}\,.
				\end{aligned}
				\end{equation}
				
				For each layer $k$, the reduced stiffness matrix is transformed to account for the layer orientation angle $\theta$ within the laminate coordinate system, yielding a reduced transformed stiffness matrix according to:
				
				\begin{equation}
					[\bar{Q}] = [T]^{-1}[Q][T]^{-T}\,,
				\end{equation}
				
				where the transformation matrix $[T]$ is defined as:
				
				\begin{equation}
				[T] =
				 \begin{bmatrix}
					 \cos^2\theta & \sin^2\theta & 2\cos\theta\sin\theta \\
					 \sin^2\theta & \cos^2\theta & -2\cos\theta\sin\theta \\
					 -\cos\theta\sin\theta & \cos\theta\sin\theta & \cos^2\theta - \sin^2\theta
				 \end{bmatrix}\,.
				\end{equation}
				
				Using the lay-ups specified for the cuticle permutations, we calculate the extensional stiffness matrix $[A]$, bending stiffness matrix $[D]$, and bending-extension coupling matrix $B$ for each laminate consisting of $n$ layers at a distance $z$ from the laminate mid-plane.
				The elements of these matrices can be found according to:
				
				\begin{equation}
					\begin{aligned}
						A_{ij} & = \sum_{k = 1}^{n}(\bar{Q}_{ij})_k(z_{k} - z_{k-1})\,, \\
						B_{ij} & = \frac{1}{2}\sum_{k = 1}^{n}(\bar{Q}_{ij})_k(z^2_{k} - z^2_{k-1})\,, \\
						D_{ij} & = \frac{1}{3}\sum_{k = 1}^{n}(\bar{Q}_{ij})_k(z^3_{k} - z^3_{k-1})\,.
					\end{aligned}
				\end{equation}
				
				These stiffness matrices relate vectors of resultant forces $\{N\}$ and bending moments $\{M\}$ to mid-surface strains and curvatures $\{\epsilon^{\degree}\}$ and $\{\kappa\}$, respectively, in the laminate according to the following relationship:
				
				\begin{equation}
					\begin{Bmatrix}
						\{N\} \\
						\{M\}
					\end{Bmatrix}
					=
					\begin{bmatrix}
						[A] & [B] \\
						[B] & [D]
					\end{bmatrix}
					\begin{Bmatrix}
					\{\epsilon^{\degree}\} \\
					\{\kappa\}
					\end{Bmatrix}\,.
				\end{equation}
				
				For symmetric laminates, $[B] = 0$, and therefore:
				
				\begin{equation}
				\label{eqn::abbd}
					\begin{aligned}
						\{N\} & = [A]\{\epsilon^{\degree}\}\,, \\
						\{M\} & = [D]\{\kappa\}\,,						
					\end{aligned}\,,
				\end{equation}
				
				or, in expanded form:
				
				\begin{equation}
					\begin{aligned}
						\begin{Bmatrix}
							N_{xx} \\
							N_{yy} \\
							N_{xy}
						\end{Bmatrix}
						& =
						\begin{bmatrix}
							A_{11} & A_{12} & A_{16} \\
							A_{21} & A_{22} & A_{26} \\
							A_{61} & A_{62} & A_{66}
						\end{bmatrix}
						\begin{Bmatrix}
							\epsilon^{\degree}_{xx} \\
							\epsilon^{\degree}_{yy} \\
							\gamma^{\degree}_{xy}
						\end{Bmatrix}\,,
						\\
						\begin{Bmatrix}
							M_{xx} \\
							M_{yy} \\
							M_{xy}
						\end{Bmatrix}
						& =
						\begin{bmatrix}
							D_{11} & D_{12} & D_{16} \\
							D_{21} & D_{22} & D_{26} \\
							D_{61} & D_{62} & D_{66}
						\end{bmatrix}
						\begin{Bmatrix}
							\kappa_{xx} \\
							\kappa_{yy} \\
							\kappa_{xy}
						\end{Bmatrix}\,.
					\end{aligned}
				\end{equation}
				
				If we make the simplifying assumptions (see ref.\info{reddy, 2007 and jones, 1998}) that \textbf{(1)} the laminate experiences pure axial loading and transverse bending (i.e., $N_{yy} = N_{xy} = 0$ and $M_{yy} = M_{xy} = 0$, respectively) and \textbf{(2)} the laminate is a beam of sufficiently high aspect ratio to minimize the Poisson effect and anisotropic shear coupling (i.e., below we effectively let $A^*_{12} = A^*_{16} = 0$ and $D^*_{12} = D^*_{16} = 0$), then we can calculate the in-plane effective flexural and axial Young's moduli of the laminate along the x-axis.
				
				For axial Young's modulus of the laminate, we first define the average membrane stresses in the laminate as:
				
				\begin{equation}
					\{\bar{\sigma}^m\} = \frac{\{N\}}{z_1 - z_n}\,.
				\end{equation}
				
				By substitution in Eq.~\ref{eqn::abbd}, we find:
				
				\begin{equation}
					\begin{aligned}
						\begin{Bmatrix}
							\bar{\sigma}^m_{xx} \\
							\bar{\sigma}^m_{yy} \\
							\bar{\tau}^m_{xy}
						\end{Bmatrix}
						& = \frac{1}{(z_1 - z_n)}
						\begin{bmatrix}
							A_{11} & A_{12} & A_{16} \\
							A_{21} & A_{22} & A_{26} \\
							A_{61} & A_{62} & A_{66}
						\end{bmatrix}
						\begin{Bmatrix}
							\epsilon^m_{xx} \\
							\epsilon^m_{yy} \\
							\gamma^m_{xy}
						\end{Bmatrix}\,,
					\end{aligned}
				\end{equation}
				
				and, by inverting this equation (let $A^* = A^{-1}$) and substituting based on the assumptions above, we infer:
				
				\begin{equation}
					\epsilon^m_{xx} = (z_1 - z_n)A^{*}_{11}\bar{\sigma}^m_{xx}\,.
				\end{equation}
				
				We therefore define Young's modulus for effective axial elasticity as:
				
				\begin{equation}
					E^m_{xx} = \frac{\bar{\sigma}^m_{xx}}{\epsilon^m_{xx}} = \frac{1}{A^*_{11}(z_1 - z_k)}
				\end{equation}
				
				To find the transverse flexural Young's modulus of the laminate, we first specify the moment-curvature relation of an Euler-Bernoulli beam:
				
				\begin{equation}
				\label{eqn::bee}
				M = EI\kappa\,.
				\end{equation}
				
				Along the x-axis, the second moment of area for a rectangular cross-section is:
				
				\begin{equation}
					I_{yy} = \frac{b(z_1 - z_n)^3}{12}\,,
				\end{equation}	
				
				Given the assumption that $M_{yy} = M_{xy} = 0$, the moment along the x-axis is related to the moment of the beam by:
				
				\begin{equation}
					M = M_{xx}b\,.
				\end{equation}
				
				Thus, given the assumption that $D_{12} = D_{16} = 0$, Young's modulus for the effective transverse flexural elasticity of the laminate can be found by making Eq.~\ref{eqn::bee} specific to transverse flexure of the x-axis and rearranging the terms:
				
				\begin{equation}
					E^m_{xx} = \frac{12M_{xx}}{(z_1 - z_n)^3\kappa_{xx}}
				\end{equation}
				
				From inversion of Eq.~\ref{eqn::abbd} (let $D^* = D^{-1}$) this reduces to:
				
				\begin{equation}
					E^m_{xx} = \frac{12}{(z_1 - z_n)^3D^*_{11}}
				\end{equation}
					
		\subsection*{Statistical analysis}
			\paragraph*{General approach}
				To explore the relationships between the composite structure and mechanical properties of the cuticle, we fit phylogenetic linear mixed-effects models to the data using maximum likelihood estimation.
				In order to control for phylogenetic non-independence in the data, we included the species of each specimen as a random effect in all models.
				We also allowed for correlation in the error term of the models, as specified by a variance-covariance matrix generated from a Brownian motion model of trait evolution along the phylogeny.
				Response variables and covariates were natural-log transformed, as needed, to ensure model residuals were normally distributed and homoscedastic.
				In all models, we tested whether the inclusion of phylogenetic correlation in the model error produced significantly better model fit, using a likelihood-ratio test and $R^{2}_{\sigma}$-difference test between the fully-specified model and a model lacking the phylogenetic effect.
			\paragraph*{Hypothesis testing}
				The following three hypotheses were tested using PGLMMs fitted using ML estimation:
				\begin{enumerate}
				\item The maximum sustained tensile force is proportional to the cross-sectional area of the endocuticle, and \emph{not} that of the exocuticle.
				\item The ultimate tensile strength of the samples is inversely proportional to the ratio of exocuticle to endocuticle at the location of fracture.
				\item Young's modulus of the samples is proportional to the length of the snout.
				\end{enumerate}
				
				We fitted a fully-specified model with the cross-sectional area of endocuticle and exocuticle at the site of fracture as fixed effects, including an interaction term, and with maximum tensile force sustained prior to fracture as a response variable.
				This model was then compared to models with only cross-sectional area of either endocuticle or exocuticle as the sole fixed effect in the model.
				We then tested the first hypothesis by using likelihood-ratio tests and $R^{2}_{\beta*}$-difference tests between each of the three models.
				
				The hypothesis that 
			
 
			\paragraph*{Model selection and fitting}
			
			\paragraph*{Estimating phylogenetic signal}
		
			
		\subsection*{Code availability}
			
		\subsection*{Data availability}

%references for methods only
	\begin{thebibliography}{50}
		\section*{References}	
			\bibitem{lamport94}
				Leslie Lamport,
				\textit{\LaTeX: a document preparation system},
				Addison Wesley, Massachusetts,
				2nd edition,
				1994.

	\end{thebibliography}
\end{document}
