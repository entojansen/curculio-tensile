\documentclass[twocolumn, linenumbers, superscriptaddress]{revtex4-1}

\usepackage{graphicx}
\usepackage{blindtext}
\usepackage{gensymb}
\usepackage{xcolor}
\usepackage[colorlinks]{hyperref}
	\hypersetup{colorlinks,
	linkcolor={red!50!black},
	citecolor={blue!60!black},
	urlcolor={blue!40!black}
	}
\usepackage{cleveref}
\graphicspath{{figures/}}

\begin{document}
	%abstract - approx. 150 words
	\begin{abstract}
		\blindtext
	\end{abstract}
	
	%title and author block
	{\title{Exoskeletal microstructure and tensile behavior of the acorn weevil rostrum}
	
	\date{\today}
	
	\author{M. Andrew Jansen}
		\email[corresponding author, email:~]{majanse1@asu.edu}
		\affiliation{School of Life Sciences, Arizona State University, Tempe, AZ 85287, USA}
	\author{Jason Williams}
		\affiliation{School for Engineering of Matter, Energy, and Transport, Arizona State University, Tempe, AZ 85287, USA}
	\author{Nikhilesh Chawla}
		\affiliation{School for Engineering of Matter, Energy, and Transport, Arizona State University, Tempe, AZ 85287, USA}
	\author{Nico M. Franz}
		\affiliation{School of Life Sciences, Arizona State University, Tempe, AZ 85287, USA}
		
	\maketitle
	}
	
%manuscript outline:

%standard intro, talk about behavior, rostrum bending, and cuticle
%weevil cuticle is known to be stronger in compression than in tension
%in Curculio this means that the greatest risk of breakage would come from tension during bending of the snout
%curculio appear to have mitigated the risk of breakage by making the rostral apex flexible
%prior work suggests that this flexibility comes from a modified composite profile in the cuticle of the rostral apex
%different parts of the head thus appear to have differently specialized cuticle to deal with extreme bending
%never any comparison across species, fractographic analysis never attempted to explain how snout is most likely to break

%we therefore set out to characterize the modifications to the cuticle of the rostral apex that presumably allow bending.
%because material rigidity and toughness are often inversely proportional, we also investigate how the composite structure of the cuticle contributes to the tensile strength of the snout, whether the flexible regions exhibit higher tensile strenth, and conduct fractographic analysis on tensile testing specimens to determine the failure modes of the rostrum under tension.
%We additionally present the results of a load cycling test for a representative specimen of a species with a highly elongate snout to explore the possibility that bending of the rostrum results in elastic deformation of the cuticle.

%This study demonstrates: (1) there are 2 microstructural modifications to the composite profile of the cuticle of the rostral apex that would enhance cuticular flexibility, (2) thickness of the encocuticle is directly proportional to the maximum tensile force sustained in tension, to the exclusion of the exocuticle, (3) the tensile strenth of the snout varies across the length of the head and is inversely proportional to exo/endocuticle ratio, such that the more flexible areas are also stronger under tension, (4) young's modulus is higher for longer snouts, reflecting increased reinforcement at the base of the snout, likely to prevent bucking fracture at the base, (5) because of the modifications to the cuticle of the rostral apex (which result in all of these changes to the mechanical properties of the snout), bending appears to be viscoelastic/elastic, with a relatively low risk of breakage during the lifetime of the beetle, (6) when the rostrum does fail, crack initiation appears to take place in the exocuticle-rich gular sutures, which would be under tension during the drilling behavior in the live insect.

%this has important implications for understanding the evolution of this highly exaggerated structure, ie suggesting negative selection of breakage instead of positive selection of flexibility. possibly a single ancestral shift for each modification, opened a large region of morphospace for exploration, resulting in present day diversity of the genus.

%-------------------------------------------------------------------------------------------------
%introduction - nature materials requires that introduction is given NO SECTION HEADER, 500 words

	\section{Microstructure of the \textit{Curculio} rostrum}\label{sec:microstructure}

		
	\section{Force-controlled loading to fracture}
	
	
	\section{Load cycling of \textit{Curculio caryae}}
	
			
	\section{Fractography of test specimens}


	\section{Conclusions} %shoot for 500 words
		
%actual methods after references
	\section{Methods}
		Methods, including statements of data availability and any associated accession codes and references, are available in the online version of this paper.
	
%50 references (as a guideline, in order of appearance in text)
	\begin{thebibliography}{50}
		\section*{References}	
			\bibitem{lamport94}
				Leslie Lamport,
				\textit{\LaTeX: a document preparation system},
				Addison Wesley, Massachusetts,
				2nd edition,
				1994.
	\end{thebibliography}

%brief and 'without effusive statements'
%thank everyone who helped with data collection, dave (SEM), charlie (specimens), paige (if I use confocal data), Sal (if I use histology/staining data or confocal data, he helped with specimen clearing).
%Financial support/ funding sources/ grants.
	\begin{acknowledgements}

	\end{acknowledgements}

%author contributions
%MAJ: did tensile tests, SEM, confocal (if applicable), data analysis.
%MAJ, NC: designed research, micro-CT (if applicable)
%MAJ, NMF: field collection of test specimens, procurement of specimen loans, specimen prep.
%MAJ, NC, NMF: All authors contributed to the writing, review, and revision of the manuscript.
	\section*{Author contributions}
		\begin{description}
		\item[Andrew Jansen] Conducted sectioning and staining, microscopy and imaging, tensile and fatigue testing, statistical analysis, and participated in manuscript preparation.
		\item[Jason Williams] Conducted tensile and fatigue testing, participated in manuscript preparation.
		\item[Nikhilesh Chawla] Facilitated microscopy, tensile and fatigue testing, and participated in manuscript preparation.
		\item[Nico Franz] Facilitated specimen acquisition and imaging, participated in manuscript preparation.
		\end{description} 
	
	\section*{Additional information}
		Supplementary information is available in the online version of the paper.
		Reprints and permissions information is available online at www.nature.com/reprints.
		Correspondence and requests for materials should be addressed to M.A.J.
	
	\section*{Competing financial interests}
		The authors declare no competing financial interests.
	
	\newpage

%provide detailed methods here, subsection for each method, statistical analysis, statements of code and data avilability.
	\section*{Methods}
		\subsection*{Histological sectioning}
			
		\subsection*{Tensile and fatigue testing}

		\subsection*{Specimen imaging and microscopy}
			
		\subsection*{Statistical analysis}
			\paragraph*{General approach}
				To explore the relationships between the composite structure and mechanical properties of the cuticle, we fit phylogenetic linear mixed-effects models to the data using maximum likelihood estimation.
				In order to control for phylogenetic non-independence in the data, we included the species of each specimen as a random effect in all models.
				We also allowed for correlation in the error term of the models, as specified by a variance-covariance matrix generated from a Brownian motion model of trait evolution along the phylogeny.
				Response variables and covariates were natural-log transformed, as needed, to ensure model residuals were normally distributed and homoscedastic.
				In all models, we tested whether the inclusion of phylogenetic correlation in the model error produced significantly better model fit, using a likelihood-ratio test and $R^{2}_{\sigma}$-difference test between the fully-specified model and a model lacking the phylogenetic effect.
			\paragraph*{Hypothesis testing}
				The following three hypotheses were tested using PGLMMs fitted using ML estimation:
				\begin{enumerate}
				\item The maximum sustained tensile force is proportional to the cross-sectional area of the endocuticle, and \emph{not} that of the exocuticle.
				\item The ultimate tensile strength of the samples is inversely proportional to the ratio of exocuticle to endocuticle at the location of fracture.
				\item Young's modulus of the samples is proportional to the length of the snout.
				\end{enumerate}
				
				We fitted a fully-specified model with the cross-sectional area of endocuticle and exocuticle at the site of fracture as fixed effects, including an interaction term, and with maximum tensile force sustained prior to fracture as a response variable.
				This model was then compared to models with only cross-sectional area of either endocuticle or exocuticle as the sole fixed effect in the model.
				We then tested the first hypothesis by using likelihood-ratio tests and $R^{2}_{\beta*}$-difference tests between each of the three models.
				
				The hypothesis that 
			
 
			\paragraph*{Model selection and fitting}
			\paragraph*{Estimating phylogenetic signal}
		
			
		\subsection*{Code availability}
			
		\subsection*{Data availability}

%references for methods only
	\begin{thebibliography}{50}
		\section*{References}	
			\bibitem{lamport94}
				Leslie Lamport,
				\textit{\LaTeX: a document preparation system},
				Addison Wesley, Massachusetts,
				2nd edition,
				1994.

	\end{thebibliography}
\end{document}