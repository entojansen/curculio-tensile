\documentclass[10pt,letterpaper]{letter}
\usepackage[utf8]{inputenc}
\usepackage{amsmath}
\usepackage{amsfonts}
\usepackage{amssymb}
\usepackage[letterpaper,left=1in, right=1in, top=1in, bottom=1in]{geometry}

\makeatletter
\let\@texttop\relax
\makeatother

\address{Andrew Jansen\\
School of Life Sciences\\
Arizona State University\\
Tempe, AZ 85287-4501} 
\signature{Andrew Jansen, PhD\\
Corresponding Author – majanse1@asu.edu\\
School of Life Sciences\\
Arizona State University\\
Tempe, AZ 85287-4501, USA} 
\begin{document} 
\begin{letter}{Xin Li\\
Associate Editor\\
xin.li@nature.com\\
}
\opening{Dear Dr. Li,} 
 
I am writing to inquire about the suitability of our manuscript entitled, ``Avoidance of catastrophic structural failure as an evolutionary constraint: Biomechanics of the acorn weevil rostrum'', for consideration for publication in Nature Materials.
The primary focus of the manuscript is the characterization of tensile and failure mechanics of the exoskeleton (a laminate composite) in a type of beetle (acorn weevils).
I believe that the methods and results we present would be an outstanding fit for your readers, while also having the potential to reach an expanded audience comprised of evolutionary biologists and biomechanists.
Our work is novel in showing that the composite microstructure of the exoskeleton of acorn weevils is locally modified to enhance both flexibility and toughness of a part of the head called the rostrum, something never before observed in insects.
In addition, we find that the modifications to the material are likely an evolutionary response to circumvent unavoidable, brittle features inherent to the anatomy the structure.

We believe that our insights are very impactful: these results are the first of their kind to conclusively demonstrate that the \emph{composite microstructure} of the exoskeleton, rather than its thickness or elastomer composition, can be optimized across species for avoidance of structural failure during extreme bending.
Our results also highlight long-ignored and under-appreciated design patterns in the composite structure of the cuticle as an essential feature of the functionality of insect exoskeletons.
This study shows how the laminate organization of the cuticle, and local variations in particular, are intimately linked to the gross morphology of macroscale structures and are of critical importance to understanding the mechanical behavior and evolution of the exoskeleton both as a material and source of biomechanical constraint upon these fascinating organisms.

I hope you will consider evaluating the manuscript to see if it falls within your journal’s scope.
Our title and abstract are copied below; I have also attached the full manuscript, in case you or another editor would like to read further.
All of the authors have agreed to submit the manuscript to Nature Materials; we feel that this journal will permit us to reach the ideal readership to disseminate our findings.
If you have any questions about the manuscript, please let me know.

Thank you very much, and we look forward to your reply.

\closing{Sincerely,}

\newpage

\cc{Jason Williams\\
Co-author\\
Research Associate\\
School for Engineering of Matter, Transport, and Energy\\
Arizona State University\\
Tempe, AZ 85287-4501, USA\\
\\
Nikhilesh Chawla\\
Co-author\\
Professor\\
School for Engineering of Matter, Transport, and Energy\\
Arizona State University\\
Tempe, AZ 85287-4501, USA\\
\\
Nico Franz\\
Co-author\\
Professor\\
School of Life Sciences\\
Arizona State University\\
Tempe, AZ 85287-4501, USA}

\bigskip

\textbf{Title:}
Avoidance of catastrophic structural failure as an evolutionary constraint: Biomechanics of the acorn weevil rostrum

\textbf{Abstract:}
The acorn weevil (\textit{Curculio} Linnaeus, 1758) rostrum (snout) exhibits remarkable flexibility and toughness derived from the microarchitecture of its exoskeleton.
Here we characterize modifications to the composite profile of the rostral cuticle that simultaneously enhance the flexibility and toughness of the distal portion of the snout.
Using Classical Laminate Plate Theory, we estimate the effect of these modifications on the elastic behavior of the exoskeleton.
We show that the tensile behavior of the rostrum across six \textit{Curculio} species with high morphological variation correlates with changes in the relative layer thicknesses and orientation angles of layers in the exoskeleton.
Accordingly, increased endocuticle thickness is strongly correlated with increased tensile strength.
Rostrum stiffness is shown to be inversely correlated with work of fracture; thus allowing a highly curved rostrum to completely straighten without structural damage.
Finally, we identify exocuticle rich invaginations of the occipital sutures both as a likely site of crack initiation in tensile failure, and as a source of morphological constraint on the evolution of the rostrum in \textit{Curculio} weevils.
We conclude that avoidance of catastrophic structural failure, as initiated in these sutures under tension, is the driving selective pressure in the evolution of the female \textit{Curculio} rostrum.
%\ps{adding a postscript} 
%\encl{list of enclosed material} 
\end{letter} 
\end{document}